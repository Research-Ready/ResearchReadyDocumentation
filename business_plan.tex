% Options for packages loaded elsewhere
% Options for packages loaded elsewhere
\PassOptionsToPackage{unicode}{hyperref}
\PassOptionsToPackage{hyphens}{url}
%
\documentclass[
  oneside,
  open=any]{scrbook}
\usepackage{xcolor}
\usepackage{amsmath,amssymb}
\setcounter{secnumdepth}{5}
\usepackage{iftex}
\ifPDFTeX
  \usepackage[T1]{fontenc}
  \usepackage[utf8]{inputenc}
  \usepackage{textcomp} % provide euro and other symbols
\else % if luatex or xetex
  \usepackage{unicode-math} % this also loads fontspec
  \defaultfontfeatures{Scale=MatchLowercase}
  \defaultfontfeatures[\rmfamily]{Ligatures=TeX,Scale=1}
\fi
\usepackage{lmodern}
\ifPDFTeX\else
  % xetex/luatex font selection
\fi
% Use upquote if available, for straight quotes in verbatim environments
\IfFileExists{upquote.sty}{\usepackage{upquote}}{}
\IfFileExists{microtype.sty}{% use microtype if available
  \usepackage[]{microtype}
  \UseMicrotypeSet[protrusion]{basicmath} % disable protrusion for tt fonts
}{}
\makeatletter
\@ifundefined{KOMAClassName}{% if non-KOMA class
  \IfFileExists{parskip.sty}{%
    \usepackage{parskip}
  }{% else
    \setlength{\parindent}{0pt}
    \setlength{\parskip}{6pt plus 2pt minus 1pt}}
}{% if KOMA class
  \KOMAoptions{parskip=half}}
\makeatother
% Make \paragraph and \subparagraph free-standing
\makeatletter
\ifx\paragraph\undefined\else
  \let\oldparagraph\paragraph
  \renewcommand{\paragraph}{
    \@ifstar
      \xxxParagraphStar
      \xxxParagraphNoStar
  }
  \newcommand{\xxxParagraphStar}[1]{\oldparagraph*{#1}\mbox{}}
  \newcommand{\xxxParagraphNoStar}[1]{\oldparagraph{#1}\mbox{}}
\fi
\ifx\subparagraph\undefined\else
  \let\oldsubparagraph\subparagraph
  \renewcommand{\subparagraph}{
    \@ifstar
      \xxxSubParagraphStar
      \xxxSubParagraphNoStar
  }
  \newcommand{\xxxSubParagraphStar}[1]{\oldsubparagraph*{#1}\mbox{}}
  \newcommand{\xxxSubParagraphNoStar}[1]{\oldsubparagraph{#1}\mbox{}}
\fi
\makeatother

\usepackage{color}
\usepackage{fancyvrb}
\newcommand{\VerbBar}{|}
\newcommand{\VERB}{\Verb[commandchars=\\\{\}]}
\DefineVerbatimEnvironment{Highlighting}{Verbatim}{commandchars=\\\{\}}
% Add ',fontsize=\small' for more characters per line
\usepackage{framed}
\definecolor{shadecolor}{RGB}{241,243,245}
\newenvironment{Shaded}{\begin{snugshade}}{\end{snugshade}}
\newcommand{\AlertTok}[1]{\textcolor[rgb]{0.68,0.00,0.00}{#1}}
\newcommand{\AnnotationTok}[1]{\textcolor[rgb]{0.37,0.37,0.37}{#1}}
\newcommand{\AttributeTok}[1]{\textcolor[rgb]{0.40,0.45,0.13}{#1}}
\newcommand{\BaseNTok}[1]{\textcolor[rgb]{0.68,0.00,0.00}{#1}}
\newcommand{\BuiltInTok}[1]{\textcolor[rgb]{0.00,0.23,0.31}{#1}}
\newcommand{\CharTok}[1]{\textcolor[rgb]{0.13,0.47,0.30}{#1}}
\newcommand{\CommentTok}[1]{\textcolor[rgb]{0.37,0.37,0.37}{#1}}
\newcommand{\CommentVarTok}[1]{\textcolor[rgb]{0.37,0.37,0.37}{\textit{#1}}}
\newcommand{\ConstantTok}[1]{\textcolor[rgb]{0.56,0.35,0.01}{#1}}
\newcommand{\ControlFlowTok}[1]{\textcolor[rgb]{0.00,0.23,0.31}{\textbf{#1}}}
\newcommand{\DataTypeTok}[1]{\textcolor[rgb]{0.68,0.00,0.00}{#1}}
\newcommand{\DecValTok}[1]{\textcolor[rgb]{0.68,0.00,0.00}{#1}}
\newcommand{\DocumentationTok}[1]{\textcolor[rgb]{0.37,0.37,0.37}{\textit{#1}}}
\newcommand{\ErrorTok}[1]{\textcolor[rgb]{0.68,0.00,0.00}{#1}}
\newcommand{\ExtensionTok}[1]{\textcolor[rgb]{0.00,0.23,0.31}{#1}}
\newcommand{\FloatTok}[1]{\textcolor[rgb]{0.68,0.00,0.00}{#1}}
\newcommand{\FunctionTok}[1]{\textcolor[rgb]{0.28,0.35,0.67}{#1}}
\newcommand{\ImportTok}[1]{\textcolor[rgb]{0.00,0.46,0.62}{#1}}
\newcommand{\InformationTok}[1]{\textcolor[rgb]{0.37,0.37,0.37}{#1}}
\newcommand{\KeywordTok}[1]{\textcolor[rgb]{0.00,0.23,0.31}{\textbf{#1}}}
\newcommand{\NormalTok}[1]{\textcolor[rgb]{0.00,0.23,0.31}{#1}}
\newcommand{\OperatorTok}[1]{\textcolor[rgb]{0.37,0.37,0.37}{#1}}
\newcommand{\OtherTok}[1]{\textcolor[rgb]{0.00,0.23,0.31}{#1}}
\newcommand{\PreprocessorTok}[1]{\textcolor[rgb]{0.68,0.00,0.00}{#1}}
\newcommand{\RegionMarkerTok}[1]{\textcolor[rgb]{0.00,0.23,0.31}{#1}}
\newcommand{\SpecialCharTok}[1]{\textcolor[rgb]{0.37,0.37,0.37}{#1}}
\newcommand{\SpecialStringTok}[1]{\textcolor[rgb]{0.13,0.47,0.30}{#1}}
\newcommand{\StringTok}[1]{\textcolor[rgb]{0.13,0.47,0.30}{#1}}
\newcommand{\VariableTok}[1]{\textcolor[rgb]{0.07,0.07,0.07}{#1}}
\newcommand{\VerbatimStringTok}[1]{\textcolor[rgb]{0.13,0.47,0.30}{#1}}
\newcommand{\WarningTok}[1]{\textcolor[rgb]{0.37,0.37,0.37}{\textit{#1}}}

\providecommand{\tightlist}{%
  \setlength{\itemsep}{0pt}\setlength{\parskip}{0pt}}\usepackage{longtable,booktabs,array}
\usepackage{calc} % for calculating minipage widths
% Correct order of tables after \paragraph or \subparagraph
\usepackage{etoolbox}
\makeatletter
\patchcmd\longtable{\par}{\if@noskipsec\mbox{}\fi\par}{}{}
\makeatother
% Allow footnotes in longtable head/foot
\IfFileExists{footnotehyper.sty}{\usepackage{footnotehyper}}{\usepackage{footnote}}
\makesavenoteenv{longtable}
\usepackage{graphicx}
\makeatletter
\newsavebox\pandoc@box
\newcommand*\pandocbounded[1]{% scales image to fit in text height/width
  \sbox\pandoc@box{#1}%
  \Gscale@div\@tempa{\textheight}{\dimexpr\ht\pandoc@box+\dp\pandoc@box\relax}%
  \Gscale@div\@tempb{\linewidth}{\wd\pandoc@box}%
  \ifdim\@tempb\p@<\@tempa\p@\let\@tempa\@tempb\fi% select the smaller of both
  \ifdim\@tempa\p@<\p@\scalebox{\@tempa}{\usebox\pandoc@box}%
  \else\usebox{\pandoc@box}%
  \fi%
}
% Set default figure placement to htbp
\def\fps@figure{htbp}
\makeatother

\makeatletter
\@ifpackageloaded{caption}{}{\usepackage{caption}}
\AtBeginDocument{%
\ifdefined\contentsname
  \renewcommand*\contentsname{Table of contents}
\else
  \newcommand\contentsname{Table of contents}
\fi
\ifdefined\listfigurename
  \renewcommand*\listfigurename{List of Figures}
\else
  \newcommand\listfigurename{List of Figures}
\fi
\ifdefined\listtablename
  \renewcommand*\listtablename{List of Tables}
\else
  \newcommand\listtablename{List of Tables}
\fi
\ifdefined\figurename
  \renewcommand*\figurename{Figure}
\else
  \newcommand\figurename{Figure}
\fi
\ifdefined\tablename
  \renewcommand*\tablename{Table}
\else
  \newcommand\tablename{Table}
\fi
}
\@ifpackageloaded{float}{}{\usepackage{float}}
\floatstyle{ruled}
\@ifundefined{c@chapter}{\newfloat{codelisting}{h}{lop}}{\newfloat{codelisting}{h}{lop}[chapter]}
\floatname{codelisting}{Listing}
\newcommand*\listoflistings{\listof{codelisting}{List of Listings}}
\makeatother
\makeatletter
\makeatother
\makeatletter
\@ifpackageloaded{caption}{}{\usepackage{caption}}
\@ifpackageloaded{subcaption}{}{\usepackage{subcaption}}
\makeatother

\usepackage{hyphenat}
\usepackage{ifthen}
\usepackage{calc}
\usepackage{calculator}

\usepackage{graphicx}
\usepackage{wallpaper}

\usepackage{geometry}

\usepackage{graphicx}
\usepackage{geometry}
\usepackage{afterpage}
\usepackage{tikz}
\usetikzlibrary{calc}
\usetikzlibrary{fadings}
\usepackage[pagecolor=none]{pagecolor}


% Set the titlepage font families







% Set the coverpage font families

\usepackage{bookmark}
\IfFileExists{xurl.sty}{\usepackage{xurl}}{} % add URL line breaks if available
\urlstyle{same}
\hypersetup{
  pdftitle={Business Plan -- Research Ready},
  pdfauthor={Christiaan Verhoef; Research Ready Team; Partner Institutions},
  hidelinks,
  pdfcreator={LaTeX via pandoc}}


\title{Business Plan -- Research Ready}
\usepackage{etoolbox}
\makeatletter
\providecommand{\subtitle}[1]{% add subtitle to \maketitle
  \apptocmd{\@title}{\par {\large #1 \par}}{}{}
}
\makeatother
\subtitle{AI Infrastructure, Automation and Reproducible Research
Systems}
\author{Christiaan Verhoef \and Research Ready Team \and Partner
Institutions}
\date{}
\begin{document}
%%%%% begin titlepage extension code

  \begin{frontmatter}

\begin{titlepage}

%%% TITLE PAGE START

% Set up alignment commands
%Page
\newcommand{\titlepagepagealign}{
\ifthenelse{\equal{left}{right}}{\raggedleft}{}
\ifthenelse{\equal{left}{center}}{\centering}{}
\ifthenelse{\equal{left}{left}}{\raggedright}{}
}


\newcommand{\titleandsubtitle}{
% Title and subtitle
{{\large{\bfseries{\nohyphens{Business Plan -- Research Ready}}}}\par
}%

\vspace{\betweentitlesubtitle}
{
{\large{\textit{\nohyphens{AI Infrastructure, Automation and
Reproducible Research Systems}}}}\par
}}
\newcommand{\titlepagetitleblock}{
\titleandsubtitle
}

\newcommand{\authorstyle}[1]{{\large{#1}}}

\newcommand{\affiliationstyle}[1]{{\large{#1}}}

\newcommand{\titlepageauthorblock}{
{\authorstyle{\nohyphens{Christiaan
Verhoef}{\textsuperscript{1}}\textsuperscript{,}{\textsuperscript{2}},  \nohyphens{Research
Ready Team}{\textsuperscript{3}} and \nohyphens{Partner
Institutions}{\textsuperscript{4}}}}}

\newcommand{\titlepageaffiliationblock}{
\hangindent=1em
\hangafter=1
{\affiliationstyle{
{1}.~Research Ready,~The Netherlands
\par\hangindent=1em\hangafter=1{2}.~Value Chain Hackers Lab,~Supply
Chain Finance Lectorate,~Windesheim University of Applied Sciences
\par\hangindent=1em\hangafter=1{3}.~Value Chain Hackers
Network,~Netherlands / Europe
\par\hangindent=1em\hangafter=1{4}.~Academic \& Industry
Collaborators,~Global


\vspace{1\baselineskip} 
}}
}
\newcommand{\headerstyled}{%
{Research Ready}
}
\newcommand{\footerstyled}{%
{\large{Research Ready\\
Tools for Applied AI \& Reproducible Research\\
https://github.com/Value-Chain-Hackers\\}}
}
\newcommand{\datestyled}{%
{}
}


\newcommand{\titlepageheaderblock}{\headerstyled}

\newcommand{\titlepagefooterblock}{
\footerstyled
}

\newcommand{\titlepagedateblock}{
\datestyled
}

%set up blocks so user can specify order
\newcommand{\titleblock}{\newlength{\betweentitlesubtitle}
\setlength{\betweentitlesubtitle}{\baselineskip}
{

{\titlepagetitleblock}
}

\vspace{4\baselineskip}
}

\newcommand{\authorblock}{{\titlepageauthorblock}

\vspace{2\baselineskip}
}

\newcommand{\affiliationblock}{{\titlepageaffiliationblock}

\vspace{1pt}
}

\newcommand{\logoblock}{{\includegraphics[width=0.25\textheight]{img/logo.png}}

\vspace{2\baselineskip}
}

\newcommand{\footerblock}{{\titlepagefooterblock}

\vspace{1pt}
}

\newcommand{\dateblock}{}

\newcommand{\headerblock}{{\titlepageheaderblock

\vspace{0pt}
}}
\newgeometry{top=3in,bottom=1in,right=1in,left=1in}
% background image
\newlength{\bgimagesize}
\setlength{\bgimagesize}{0.5\paperwidth}
\LENGTHDIVIDE{\bgimagesize}{\paperwidth}{\theRatio} % from calculator pkg
\ThisULCornerWallPaper{\theRatio}{img/corner-bg.png}

\thispagestyle{empty} % no page numbers on titlepages


\newcommand{\vrulecode}{\textcolor{black}{\rule{\vrulewidth}{\textheight}}}
\newlength{\vrulewidth}
\setlength{\vrulewidth}{1pt}
\newlength{\B}
\setlength{\B}{\ifdim\vrulewidth > 0pt 0.05\textwidth\else 0pt\fi}
\newlength{\minipagewidth}
\ifthenelse{\equal{left}{left} \OR \equal{left}{right} }
{% True case
\setlength{\minipagewidth}{\textwidth - \vrulewidth - \B - 0.1\textwidth}
}{
\setlength{\minipagewidth}{\textwidth - 2\vrulewidth - 2\B - 0.1\textwidth}
}
\ifthenelse{\equal{left}{left} \OR \equal{left}{leftright}}
{% True case
\raggedleft % needed for the minipage to work
\vrulecode
\hspace{\B}
}{%
\raggedright % else it is right only and width is not 0
}
% [position of box][box height][inner position]{width}
% [s] means stretch out vertically; assuming there is a vfill
\begin{minipage}[b][\textheight][s]{\minipagewidth}
\titlepagepagealign
\titleblock

\authorblock

\affiliationblock

\vfill

\logoblock

\footerblock
\par

\end{minipage}\ifthenelse{\equal{left}{right} \OR \equal{left}{leftright} }{
\hspace{\B}
\vrulecode}{}
\clearpage
\restoregeometry
%%% TITLE PAGE END
\end{titlepage}
\setcounter{page}{1}
\end{frontmatter}

%%%%% end titlepage extension code

\renewcommand*\contentsname{Table of contents}
{
\setcounter{tocdepth}{2}
\tableofcontents
}
\listoffigures
\listoftables

\mainmatter
\chapter{Business Plan}\label{business-plan}

\chapter{1. Executive Summary}\label{executive-summary}

\section{1.1 Purpose of the Business
Plan}\label{purpose-of-the-business-plan}

A short overview of the company, strategic intent, and expected
outcomes.

\section{1.2 Vision and Mission}\label{vision-and-mission}

\begin{itemize}
\tightlist
\item
  \textbf{Vision:}\\
  A concise statement describing what long-term impact the company aims
  to create.\\
\item
  \textbf{Mission:}\\
  How the company operates today to realise that vision.
\end{itemize}

\section{1.3 Strategic Highlights}\label{strategic-highlights}

Provide 4 to 6 high-level strategic messages.

\textbf{Example:} - Strong market demand in research automation and AI
infrastructure.\\
- Opportunity to modernise research practices at scale.\\
- Diversified revenue model.\\
- Strong early traction with institutional partners.

\section{1.4 Financial Snapshot
(High-Level)}\label{financial-snapshot-high-level}

Create placeholders for revenue, cost base, and profitability
projections.

\begin{longtable}[]{@{}llll@{}}
\toprule\noalign{}
Metric & Year 1 & Year 3 & Notes \\
\midrule\noalign{}
\endhead
\bottomrule\noalign{}
\endlastfoot
Revenue & & & \\
Gross Margin & & & \\
EBITDA & & & \\
Investment Need & & & \\
\end{longtable}

\chapter{2. Problem Definition}\label{problem-definition}

\section{2.1 Core Market Problem}\label{core-market-problem}

Describe the structural problem in the industry.

\textbf{Prompts to fill:} - What friction or inefficiency is common in
the target market?\\
- What is the financial or operational cost of the problem?\\
- Who is affected? Why now?

\section{2.2 Evidence of the Problem}\label{evidence-of-the-problem}

Describe data or facts that prove the problem exists.

Examples: - Time wasted in manual workflows\\
- Confusion due to fragmented tools\\
- Lack of reproducibility in research\\
- Skills gap for AI adoption

\section{2.3 Impact of Inaction}\label{impact-of-inaction}

What happens if no solution is implemented?

\begin{longtable}[]{@{}ll@{}}
\toprule\noalign{}
Category & Impact \\
\midrule\noalign{}
\endhead
\bottomrule\noalign{}
\endlastfoot
Financial & Lost revenue, higher cost \\
Operational & Slower projects, bottlenecks \\
Strategic & Inability to scale or innovate \\
Compliance & Rising pressure from regulations \\
\end{longtable}

\section{2.4 Stakeholders Affected}\label{stakeholders-affected}

List groups impacted by the problem.

\begin{itemize}
\tightlist
\item
  Research teams\\
\item
  Educational institutions\\
\item
  Industry partners\\
\item
  Students and trainees\\
\item
  Compliance officers
\end{itemize}

\chapter{3. Market Context}\label{market-context}

\section{3.1 Market Size and Growth}\label{market-size-and-growth}

Add placeholders for TAM, SAM, SOM.

\begin{longtable}[]{@{}lll@{}}
\toprule\noalign{}
Market Layer & Value & Notes \\
\midrule\noalign{}
\endhead
\bottomrule\noalign{}
\endlastfoot
Total Addressable Market (TAM) & & Sector-wide demand \\
Serviceable Available Market (SAM) & & Region or segment \\
Serviceable Obtainable Market (SOM) & & Practical share \\
\end{longtable}

\section{3.2 Industry Trends}\label{industry-trends}

Identify 4 to 6 key trends affecting this market.

Examples: - Shift toward AI-assisted research\\
- Increased demand for automation\\
- Rising interest in data reproducibility\\
- Growth in digital transformation budgets

\section{3.3 Customer Demand Drivers}\label{customer-demand-drivers}

Describe forces increasing demand for your solution.

\begin{itemize}
\tightlist
\item
  Academic teams pressured to produce more with fewer resources\\
\item
  Compliance requirements (CSRD, CSDDD, data transparency)\\
\item
  Students expecting AI-first learning environments\\
\item
  Institutions wanting in-house AI capability
\end{itemize}

\section{3.4 Competitive and Substitute
Landscape}\label{competitive-and-substitute-landscape}

Provide a high-level view of competitors and alternative solutions.

\begin{longtable}[]{@{}llll@{}}
\toprule\noalign{}
Competitor Type & Example & Strengths & Weaknesses \\
\midrule\noalign{}
\endhead
\bottomrule\noalign{}
\endlastfoot
Direct Competitors & & & \\
Indirect Competitors & & & \\
Internal Solutions & & & \\
\end{longtable}

\section{3.5 Barriers to Entry}\label{barriers-to-entry}

What makes this market difficult for new entrants?

\begin{itemize}
\tightlist
\item
  Technical complexity\\
\item
  Need for multidisciplinary expertise\\
\item
  Trust and credibility requirements in education and research\\
\item
  Integration challenges
\end{itemize}

\section{3.6 Regulatory and Policy
Landscape}\label{regulatory-and-policy-landscape}

If relevant, list the regulatory frameworks that influence the market.

Examples: - GDPR\\
- CSRD\\
- Digital transformation mandates in education

\chapter{4. Solution Overview}\label{solution-overview}

\section{4.1 Solution Definition}\label{solution-definition}

A concise description of the solution and its purpose.

\textbf{Prompts to complete:} - What is the core solution?\\
- What measurable outcome does it create?\\
- How does it resolve the stated problem?\\
- Who benefits directly and indirectly?

\section{4.2 Solution Architecture
(High-Level)}\label{solution-architecture-high-level}

Describe the macro-level structure of the solution.

Component categories: - Input layer\\
- Processing and automation layer\\
- AI inference or analytics layer\\
- Storage and data models\\
- Interfaces and outputs\\
- Integrations\\
- Deployment model

\textbf{Placeholder diagram (replace with mermaid or image if needed):}

\begin{Shaded}
\begin{Highlighting}[]
\NormalTok{flowchart LR}
\NormalTok{    A[User Inputs] {-}{-}\textgreater{} B[Interface Layer\textless{}br/\textgreater{}(Web, Desktop, API)]}
\NormalTok{    B {-}{-}\textgreater{} C[Processing \& Automation Layer\textless{}br/\textgreater{}(Pipelines, Validation, Scheduling)]}
\NormalTok{    C {-}{-}\textgreater{} D[AI \& Analytics Layer\textless{}br/\textgreater{}(Models, Rules, NLP, Summaries)]}
\NormalTok{    D {-}{-}\textgreater{} E[Data Layer\textless{}br/\textgreater{}(Databases, File Storage, Metadata)]}
\NormalTok{    E {-}{-}\textgreater{} F[Output Layer\textless{}br/\textgreater{}(Reports, Dashboards, Exports)]}
\NormalTok{    C {-}{-}\textgreater{} G[Integrations\textless{}br/\textgreater{}(External APIs, LMS, Research Tools)]}
\end{Highlighting}
\end{Shaded}

\section{4.3 Key Features and
Capabilities}\label{key-features-and-capabilities}

\begin{longtable}[]{@{}
  >{\raggedright\arraybackslash}p{(\linewidth - 4\tabcolsep) * \real{0.2250}}
  >{\raggedright\arraybackslash}p{(\linewidth - 4\tabcolsep) * \real{0.3250}}
  >{\raggedright\arraybackslash}p{(\linewidth - 4\tabcolsep) * \real{0.4500}}@{}}
\toprule\noalign{}
\begin{minipage}[b]{\linewidth}\raggedright
Feature
\end{minipage} & \begin{minipage}[b]{\linewidth}\raggedright
Description
\end{minipage} & \begin{minipage}[b]{\linewidth}\raggedright
Value Delivered
\end{minipage} \\
\midrule\noalign{}
\endhead
\bottomrule\noalign{}
\endlastfoot
Workflow Automation & Automatically processes routines & Efficiency,
consistency \\
AI-supported Insights & Summarisation, pattern detection & Better
decisions \\
Reporting Engine & Generates reproducible outputs & Higher quality
research \\
Integrations & Connects with existing systems & Easy adoption \\
Security \& Compliance & Safe data processing & Reduced risk \\
\end{longtable}

\section{4.4 User Workflow}\label{user-workflow}

\textbf{Example structure (customise as needed):} 1. User logs in
through a web or desktop interface\\
2. User uploads data or connects a datasource\\
3. System validates and pre-processes the data\\
4. Automated workflows run analyses or transformations\\
5. AI module generates insights or summaries\\
6. User receives outputs (reports, dashboards, models)\\
7. Feedback loop improves future predictions

\begin{Shaded}
\begin{Highlighting}[]
\NormalTok{flowchart TD}
\NormalTok{    A[Login / Access] {-}{-}\textgreater{} B[Upload Data\textless{}br/\textgreater{}(or connect datasource)]}
\NormalTok{    B {-}{-}\textgreater{} C[Data Validation\textless{}br/\textgreater{}(format, completeness)]}
\NormalTok{    C {-}{-}\textgreater{} D[Automated Processing\textless{}br/\textgreater{}(workflows, transformations)]}
\NormalTok{    D {-}{-}\textgreater{} E[AI Analysis\textless{}br/\textgreater{}(summaries, extraction, insights)]}
\NormalTok{    E {-}{-}\textgreater{} F[Output Generation\textless{}br/\textgreater{}(reports, dashboards, exports)]}
\NormalTok{    F {-}{-}\textgreater{} G[Review \& Feedback Loop\textless{}br/\textgreater{}(improves future accuracy)]}
\end{Highlighting}
\end{Shaded}

\section{4.5 Expected Outcomes}\label{expected-outcomes}

\begin{longtable}[]{@{}lll@{}}
\toprule\noalign{}
Outcome Category & KPI & Expected Improvement \\
\midrule\noalign{}
\endhead
\bottomrule\noalign{}
\endlastfoot
Productivity & Hours per process & Reduction of 40--70 percent \\
Quality & Error rate & Reduction to below 5 percent \\
Insight & Time to insights & 2x to 5x acceleration \\
Compliance & Traceability completeness & Significant improvement \\
Cost & Operational cost & Lowered through automation \\
\end{longtable}

\section{4.6 Solution Validation and
Evidence}\label{solution-validation-and-evidence}

Use early tests, prototypes, or simulated performance.\\
R analysis helps quantify expected improvements.

\begin{Shaded}
\begin{Highlighting}[]
\NormalTok{flowchart TD}
\NormalTok{    A[Input Data\textless{}br/\textgreater{}(raw research material)] {-}{-}\textgreater{} B[Preprocessing\textless{}br/\textgreater{}(cleaning, structuring)]}
\NormalTok{    B {-}{-}\textgreater{} C[Automation Engine\textless{}br/\textgreater{}(rules, workflows)]}
\NormalTok{    C {-}{-}\textgreater{} D[AI Models\textless{}br/\textgreater{}(summaries, predictions, extraction)]}
\NormalTok{    D {-}{-}\textgreater{} E[Evaluation Layer\textless{}br/\textgreater{}(quality checks, manual review)]}
\NormalTok{    E {-}{-}\textgreater{} F[Metrics\textless{}br/\textgreater{}(accuracy, time saved, error rates)]}
\NormalTok{    F {-}{-}\textgreater{} G[Validation Report\textless{}br/\textgreater{}(evidence, benchmarks, outputs)]}
\end{Highlighting}
\end{Shaded}

\section{4.7 Differentiators}\label{differentiators}

Possible categories:

\begin{itemize}
\tightlist
\item
  Faster deployment than alternatives
\item
  Works with existing tools
\item
  Built with modular architecture
\item
  Designed for non-technical users
\item
  High level of automation
\item
  Integration of statistical analysis, AI, and reporting in one pipeline
\end{itemize}

\section{4.8 Technical Requirements and
Dependencies}\label{technical-requirements-and-dependencies}

List the components needed for deployment.

Suggested structure:

\begin{itemize}
\tightlist
\item
  OS requirements
\item
  Hardware and compute needs
\item
  Required runtimes (R, Python, Docker, etc)
\item
  Network configuration
\item
  Integration endpoints (APIs, databases)
\item
  Storage requirements
\end{itemize}

\section{4.9 KPIs for Measuring
Success}\label{kpis-for-measuring-success}

\begin{longtable}[]{@{}
  >{\raggedright\arraybackslash}p{(\linewidth - 4\tabcolsep) * \real{0.3385}}
  >{\raggedright\arraybackslash}p{(\linewidth - 4\tabcolsep) * \real{0.3538}}
  >{\raggedright\arraybackslash}p{(\linewidth - 4\tabcolsep) * \real{0.3077}}@{}}
\toprule\noalign{}
\begin{minipage}[b]{\linewidth}\raggedright
KPI Category
\end{minipage} & \begin{minipage}[b]{\linewidth}\raggedright
KPI Definition
\end{minipage} & \begin{minipage}[b]{\linewidth}\raggedright
Target Benchmark
\end{minipage} \\
\midrule\noalign{}
\endhead
\bottomrule\noalign{}
\endlastfoot
Adoption & Monthly active users & X \\
Operational Efficiency & Average cycle time & Y percent reduction \\
Accuracy & Output error rate & \textless{} Z percent \\
Performance & Dataset processing time & X seconds or minutes \\
Satisfaction & User feedback / NPS & 40+ \\
\end{longtable}

\begin{Shaded}
\begin{Highlighting}[]
\NormalTok{flowchart LR}
\NormalTok{    A[KPIs] {-}{-}\textgreater{} B[Adoption]}
\NormalTok{    A {-}{-}\textgreater{} C[Operational Efficiency]}
\NormalTok{    A {-}{-}\textgreater{} D[Accuracy]}
\NormalTok{    A {-}{-}\textgreater{} E[Performance]}
\NormalTok{    A {-}{-}\textgreater{} F[Satisfaction]}

\NormalTok{    B {-}{-}\textgreater{} B1[Monthly Active Users]}
\NormalTok{    C {-}{-}\textgreater{} C1[Cycle Time Reduction]}
\NormalTok{    D {-}{-}\textgreater{} D1[Error Rate]}
\NormalTok{    E {-}{-}\textgreater{} E1[Processing Time]}
\NormalTok{    F {-}{-}\textgreater{} F1[NPS / Feedback Scores]}
\end{Highlighting}
\end{Shaded}

\section{4.10 Future Enhancements}\label{future-enhancements}

\begin{longtable}[]{@{}
  >{\raggedright\arraybackslash}p{(\linewidth - 4\tabcolsep) * \real{0.2532}}
  >{\raggedright\arraybackslash}p{(\linewidth - 4\tabcolsep) * \real{0.4051}}
  >{\raggedright\arraybackslash}p{(\linewidth - 4\tabcolsep) * \real{0.3418}}@{}}
\toprule\noalign{}
\begin{minipage}[b]{\linewidth}\raggedright
Enhancement
\end{minipage} & \begin{minipage}[b]{\linewidth}\raggedright
Description
\end{minipage} & \begin{minipage}[b]{\linewidth}\raggedright
Expected Value
\end{minipage} \\
\midrule\noalign{}
\endhead
\bottomrule\noalign{}
\endlastfoot
Predictive Analytics & ML-driven forecasting & Insight improvements \\
New Integrations & More data connectors & Easier onboarding \\
Advanced Automation & Multi-step intelligent workflows & Higher
efficiency \\
Visual Dashboards & Interactive insight layers & Better
decision-making \\
Scalability Upgrades & Horizontal or vertical scaling & Support for
larger datasets \\
\end{longtable}

\begin{Shaded}
\begin{Highlighting}[]
\NormalTok{flowchart TD}
\NormalTok{    A[Current System] {-}{-}\textgreater{} B[Predictive Analytics]}
\NormalTok{    A {-}{-}\textgreater{} C[New Integrations]}
\NormalTok{    A {-}{-}\textgreater{} D[Advanced Automation]}
\NormalTok{    A {-}{-}\textgreater{} E[Visual Dashboards]}
\NormalTok{    A {-}{-}\textgreater{} F[Scalability Upgrades]}

\NormalTok{    B {-}{-}\textgreater{} B1[Forecasting Models]}
\NormalTok{    C {-}{-}\textgreater{} C1[Additional Data Connectors]}
\NormalTok{    D {-}{-}\textgreater{} D1[Multi{-}step Pipelines]}
\NormalTok{    E {-}{-}\textgreater{} E1[Interactive Views]}
\NormalTok{    F {-}{-}\textgreater{} F1[Higher Dataset Capacity]}
\end{Highlighting}
\end{Shaded}

\chapter{5. Unique Value Proposition}\label{unique-value-proposition}

\section{5.1 Core Value Proposition}\label{core-value-proposition}

A concise statement describing the unique value delivered to the
customer.

\textbf{Prompts to complete:} - What transformation does the customer
experience?\\
- What measurable outcome improves?\\
- Why is this meaningfully better than alternatives?

\section{5.2 Value Drivers}\label{value-drivers}

Describe the core elements that create value.

\begin{longtable}[]{@{}
  >{\raggedright\arraybackslash}p{(\linewidth - 4\tabcolsep) * \real{0.3111}}
  >{\raggedright\arraybackslash}p{(\linewidth - 4\tabcolsep) * \real{0.2889}}
  >{\raggedright\arraybackslash}p{(\linewidth - 4\tabcolsep) * \real{0.4000}}@{}}
\toprule\noalign{}
\begin{minipage}[b]{\linewidth}\raggedright
Value Driver
\end{minipage} & \begin{minipage}[b]{\linewidth}\raggedright
Description
\end{minipage} & \begin{minipage}[b]{\linewidth}\raggedright
Customer Impact
\end{minipage} \\
\midrule\noalign{}
\endhead
\bottomrule\noalign{}
\endlastfoot
Speed & Accelerates key workflows & Faster delivery, higher
throughput \\
Automation & Removes manual steps & Lower operational cost \\
Accuracy & Reduces human error & Higher data integrity \\
Integration & Fits into existing environments & Low switching cost \\
Intelligence & AI-supported insights & Better decisions \\
\end{longtable}

\section{5.3 Customer Pain Relief}\label{customer-pain-relief}

Identify the pains your solution eliminates.

\begin{longtable}[]{@{}ll@{}}
\toprule\noalign{}
Pain Point & How the Solution Addresses It \\
\midrule\noalign{}
\endhead
\bottomrule\noalign{}
\endlastfoot
Manual processes & Automated workflows \\
Fragmented tools & Unified ecosystem \\
Slow reporting & Instant reproducible outputs \\
Skill gaps & Guided AI support \\
\end{longtable}

\section{5.4 Unique Strengths vs
Competitors}\label{unique-strengths-vs-competitors}

Highlight differentiators using a simple comparison matrix.

\begin{longtable}[]{@{}llll@{}}
\toprule\noalign{}
Dimension & Your Solution & Alternative A & Alternative B \\
\midrule\noalign{}
\endhead
\bottomrule\noalign{}
\endlastfoot
Setup Time & & & \\
Ease of Use & & & \\
Automation Depth & & & \\
AI Integration & & & \\
Total Cost & & & \\
\end{longtable}

\section{5.5 Evidence of Value}\label{evidence-of-value}

Placeholder for pilot data, testimonials, or performance benchmarks.

\section{5.6 Why Now}\label{why-now}

Describe the strategic urgency.

Prompts:

\begin{itemize}
\tightlist
\item
  What market shift makes this moment ideal?
\item
  What technology maturity enables this capability?
\item
  What regulatory pressures increase demand?
\end{itemize}

\section{5.7 Proof of Differentiation}\label{proof-of-differentiation}

Describe any proprietary elements.

Examples:

\begin{itemize}
\tightlist
\item
  Deployment methodology
\item
  Integrated toolchain
\item
  Unique automations
\item
  Data models
\item
  Domain-specialised workflows
\end{itemize}

\section{5.8 Value Proposition Statement
Template}\label{value-proposition-statement-template}

``For \textbf{{[}target customer{]}} who \textbf{{[}core problem{]}},
our solution \textbf{{[}solution category{]}} provides \textbf{{[}core
value{]}}, resulting in \textbf{{[}key outcome{]}}, unlike
\textbf{{[}alternatives{]}}, we \textbf{{[}unique differentiator{]}}.''

\chapter{6. Business Model}\label{business-model}

\section{6.1 Overview of the Business
Model}\label{overview-of-the-business-model}

A clear explanation of how the organisation creates, delivers, and
captures value.

Prompts to complete:\\
- What value is being created for the customer?\\
- How is this value delivered operationally?\\
- How does the business earn revenue from this value?\\
- What structural advantages does the model create?

\begin{Shaded}
\begin{Highlighting}[]
\NormalTok{flowchart LR}
\NormalTok{    A[Value Creation] {-}{-}\textgreater{} B[Value Delivery]}
\NormalTok{    B {-}{-}\textgreater{} C[Value Capture]}

\NormalTok{    A {-}{-}\textgreater{} A1[Automation \& AI Capabilities]}
\NormalTok{    A {-}{-}\textgreater{} A2[Integrated Research Workflows]}

\NormalTok{    B {-}{-}\textgreater{} B1[Platforms \& Tools]}
\NormalTok{    B {-}{-}\textgreater{} B2[Professional Services]}
\NormalTok{    B {-}{-}\textgreater{} B3[Training \& Support]}

\NormalTok{    C {-}{-}\textgreater{} C1[Subscriptions]}
\NormalTok{    C {-}{-}\textgreater{} C2[Project Revenue]}
\NormalTok{    C {-}{-}\textgreater{} C3[Support Contracts]}
\end{Highlighting}
\end{Shaded}

\section{6.2 Revenue Streams}\label{revenue-streams}

List and describe each way the business generates income.

\begin{longtable}[]{@{}
  >{\raggedright\arraybackslash}p{(\linewidth - 6\tabcolsep) * \real{0.2807}}
  >{\raggedright\arraybackslash}p{(\linewidth - 6\tabcolsep) * \real{0.2281}}
  >{\raggedright\arraybackslash}p{(\linewidth - 6\tabcolsep) * \real{0.2807}}
  >{\raggedright\arraybackslash}p{(\linewidth - 6\tabcolsep) * \real{0.2105}}@{}}
\toprule\noalign{}
\begin{minipage}[b]{\linewidth}\raggedright
Revenue Stream
\end{minipage} & \begin{minipage}[b]{\linewidth}\raggedright
Description
\end{minipage} & \begin{minipage}[b]{\linewidth}\raggedright
Pricing Logic
\end{minipage} & \begin{minipage}[b]{\linewidth}\raggedright
Recurrence
\end{minipage} \\
\midrule\noalign{}
\endhead
\bottomrule\noalign{}
\endlastfoot
Subscriptions & Access to tools or platforms & Per month or year &
Recurring \\
Professional Services & Implementation and setup & Fixed project fee &
One-time \\
Training \& Workshops & Instructor-led or on-demand training & Per
session or per seat & Occasional \\
Custom Development & Bespoke features or integrations & Project-based &
One-time \\
Support Contracts & Priority support and SLAs & Tiered plans &
Recurring \\
\end{longtable}

\begin{Shaded}
\begin{Highlighting}[]
\NormalTok{flowchart TD}
\NormalTok{    A[Revenue Streams] {-}{-}\textgreater{} B[Subscriptions]}
\NormalTok{    A {-}{-}\textgreater{} C[Professional Services]}
\NormalTok{    A {-}{-}\textgreater{} D[Training \& Workshops]}
\NormalTok{    A {-}{-}\textgreater{} E[Custom Development]}
\NormalTok{    A {-}{-}\textgreater{} F[Support Contracts]}

\NormalTok{    B {-}{-}\textgreater{} B1[Recurring Revenue]}
\NormalTok{    C {-}{-}\textgreater{} C1[Implementation Fees]}
\NormalTok{    D {-}{-}\textgreater{} D1[Per{-}seat or Session Fees]}
\NormalTok{    E {-}{-}\textgreater{} E1[Project{-}based Fees]}
\NormalTok{    F {-}{-}\textgreater{} F1[Tiered Support Plans]}
\end{Highlighting}
\end{Shaded}

\section{6.3 Pricing Model}\label{pricing-model}

Describe the pricing strategy and structure.

Prompts:\\
- What pricing strategy is used (value-based, competitive, tiered)?\\
- Are there customer segments with different pricing expectations?\\
- How does pricing scale with usage, seats, data volume, or complexity?

Example table:

\begin{longtable}[]{@{}llll@{}}
\toprule\noalign{}
Tier & Target Customer & Included Features & Price \\
\midrule\noalign{}
\endhead
\bottomrule\noalign{}
\endlastfoot
Basic & Small teams & Core features & \\
Pro & Medium organisations & Advanced automation + integrations & \\
Enterprise & Large institutions & Full stack + support & \\
\end{longtable}

\begin{Shaded}
\begin{Highlighting}[]
\NormalTok{flowchart LR}
\NormalTok{    A[Pricing Model] {-}{-}\textgreater{} B[Tiers]}
\NormalTok{    A {-}{-}\textgreater{} C[Value Metrics]}
\NormalTok{    A {-}{-}\textgreater{} D[Scaling Rules]}

\NormalTok{    B {-}{-}\textgreater{} B1[Basic]}
\NormalTok{    B {-}{-}\textgreater{} B2[Pro]}
\NormalTok{    B {-}{-}\textgreater{} B3[Enterprise]}

\NormalTok{    C {-}{-}\textgreater{} C1[Seats]}
\NormalTok{    C {-}{-}\textgreater{} C2[Usage Volume]}
\NormalTok{    C {-}{-}\textgreater{} C3[Feature Access]}

\NormalTok{    D {-}{-}\textgreater{} D1[Add{-}ons]}
\NormalTok{    D {-}{-}\textgreater{} D2[Storage or Compute Scaling]}
\NormalTok{    D {-}{-}\textgreater{} D3[Integration Complexity]}
\end{Highlighting}
\end{Shaded}

\section{6.4 Cost Structure}\label{cost-structure}

Identify the major cost categories of the business.

\begin{longtable}[]{@{}lll@{}}
\toprule\noalign{}
Cost Category & Description & Cost Type \\
\midrule\noalign{}
\endhead
\bottomrule\noalign{}
\endlastfoot
Infrastructure & Hosting, compute, storage & Fixed / Variable \\
Development & Engineering, updates, roadmap & Fixed \\
Support & Customer and technical support & Variable \\
Sales \& Marketing & Outreach, materials, ads & Variable \\
Operations & Admin, legal, internal tooling & Fixed \\
\end{longtable}

\begin{Shaded}
\begin{Highlighting}[]
\NormalTok{flowchart TD}
\NormalTok{    A[Cost Structure] {-}{-}\textgreater{} B[Infrastructure]}
\NormalTok{    A {-}{-}\textgreater{} C[Development]}
\NormalTok{    A {-}{-}\textgreater{} D[Support]}
\NormalTok{    A {-}{-}\textgreater{} E[Sales \& Marketing]}
\NormalTok{    A {-}{-}\textgreater{} F[Operations]}

\NormalTok{    B {-}{-}\textgreater{} B1[Hosting\textless{}br/\textgreater{}Compute\textless{}br/\textgreater{}Storage]}
\NormalTok{    C {-}{-}\textgreater{} C1[Engineering\textless{}br/\textgreater{}Feature Roadmap]}
\NormalTok{    D {-}{-}\textgreater{} D1[Customer Support\textless{}br/\textgreater{}Technical Assistance]}
\NormalTok{    E {-}{-}\textgreater{} E1[Ads\textless{}br/\textgreater{}Outreach\textless{}br/\textgreater{}Materials]}
\NormalTok{    F {-}{-}\textgreater{} F1[Admin\textless{}br/\textgreater{}Legal\textless{}br/\textgreater{}Internal Tools]}
\end{Highlighting}
\end{Shaded}

\section{6.5 Customer Acquisition
Model}\label{customer-acquisition-model}

Explain how customers discover, evaluate, and adopt the solution.

Prompts:\\
- What channels bring customers into the funnel?\\
- What is the step-by-step process from lead to conversion?\\
- What materials, demos, or touchpoints support the decision?\\
- What is the follow-up or nurturing plan?

Example structure:

\begin{longtable}[]{@{}lll@{}}
\toprule\noalign{}
Funnel Stage & Activity & KPI \\
\midrule\noalign{}
\endhead
\bottomrule\noalign{}
\endlastfoot
Awareness & LinkedIn, leads, webinars & Impressions \\
Consideration & Demo, consultation, proposals & Conversion rate \\
Decision & Contracting & Close rate \\
Onboarding & Setup and training & Activation \% \\
\end{longtable}

\begin{Shaded}
\begin{Highlighting}[]
\NormalTok{flowchart TD}
\NormalTok{    A[Awareness] {-}{-}\textgreater{} B[Consideration]}
\NormalTok{    B {-}{-}\textgreater{} C[Decision]}
\NormalTok{    C {-}{-}\textgreater{} D[Onboarding]}

\NormalTok{    A {-}{-}\textgreater{} A1[LinkedIn\textless{}br/\textgreater{}Webinars\textless{}br/\textgreater{}Partnerships]}
\NormalTok{    B {-}{-}\textgreater{} B1[Demos\textless{}br/\textgreater{}Consultations\textless{}br/\textgreater{}Proposals]}
\NormalTok{    C {-}{-}\textgreater{} C1[Contracting\textless{}br/\textgreater{}Negotiation]}
\NormalTok{    D {-}{-}\textgreater{} D1[Setup\textless{}br/\textgreater{}Training\textless{}br/\textgreater{}Activation]}
\end{Highlighting}
\end{Shaded}

\section{6.6 Delivery and Value
Capture}\label{delivery-and-value-capture}

Describe how the value is operationally delivered.

Prompts:\\
- What delivery model is used (remote, hybrid, onsite)?\\
- What steps occur from kickoff to final delivery?\\
- How is customer satisfaction ensured?\\
- How does the business retain customers?

\begin{Shaded}
\begin{Highlighting}[]
\NormalTok{flowchart LR}
\NormalTok{    A[Project Kickoff] {-}{-}\textgreater{} B[Configuration]}
\NormalTok{    B {-}{-}\textgreater{} C[Validation]}
\NormalTok{    C {-}{-}\textgreater{} D[Handover]}
\NormalTok{    D {-}{-}\textgreater{} E[Ongoing Value Capture]}

\NormalTok{    A {-}{-}\textgreater{} A1[Scope\textless{}br/\textgreater{}Alignment]}
\NormalTok{    B {-}{-}\textgreater{} B1[Technical Setup\textless{}br/\textgreater{}Integrations]}
\NormalTok{    C {-}{-}\textgreater{} C1[Testing\textless{}br/\textgreater{}Feedback]}
\NormalTok{    D {-}{-}\textgreater{} D1[Training\textless{}br/\textgreater{}Documentation]}
\NormalTok{    E {-}{-}\textgreater{} E1[Support\textless{}br/\textgreater{}Renewals\textless{}br/\textgreater{}Upsell Paths]}
\end{Highlighting}
\end{Shaded}

\section{6.7 Scalability Factors}\label{scalability-factors}

Explain what allows the business to scale and what might limit scale.

\begin{longtable}[]{@{}lll@{}}
\toprule\noalign{}
Factor & Positive Impact on Scale & Risks / Constraints \\
\midrule\noalign{}
\endhead
\bottomrule\noalign{}
\endlastfoot
Automation & Reduces marginal cost & High initial setup \\
Modular Architecture & Easier to expand & Requires governance \\
Cloud Delivery & Global reach & Compliance dependency \\
Partner Ecosystem & Amplifies adoption & Quality control needed \\
\end{longtable}

\begin{Shaded}
\begin{Highlighting}[]
\NormalTok{flowchart TD}
\NormalTok{    A[Scalability Factors] {-}{-}\textgreater{} B[Automation]}
\NormalTok{    A {-}{-}\textgreater{} C[Modular Architecture]}
\NormalTok{    A {-}{-}\textgreater{} D[Cloud Delivery]}
\NormalTok{    A {-}{-}\textgreater{} E[Partner Ecosystem]}

\NormalTok{    B {-}{-}\textgreater{} B1[Lower Marginal Cost]}
\NormalTok{    C {-}{-}\textgreater{} C1[Extend Without Rewrites]}
\NormalTok{    D {-}{-}\textgreater{} D1[Global Reach\textless{}br/\textgreater{}Elastic Resources]}
\NormalTok{    E {-}{-}\textgreater{} E1[Distributed Delivery\textless{}br/\textgreater{}Faster Adoption]}
\end{Highlighting}
\end{Shaded}

\section{6.8 Key Partnerships}\label{key-partnerships}

List the external parties required to operate effectively.

Examples: - Cloud or infrastructure providers\\
- AI or analytics vendors\\
- Technology partners\\
- Academic collaborators\\
- Channel partners

\begin{Shaded}
\begin{Highlighting}[]
\NormalTok{flowchart LR}
\NormalTok{    A[Key Partnerships] {-}{-}\textgreater{} B[Infrastructure Providers]}
\NormalTok{    A {-}{-}\textgreater{} C[AI \& Analytics Vendors]}
\NormalTok{    A {-}{-}\textgreater{} D[Technology Partners]}
\NormalTok{    A {-}{-}\textgreater{} E[Academic Collaborators]}
\NormalTok{    A {-}{-}\textgreater{} F[Channel Partners]}

\NormalTok{    B {-}{-}\textgreater{} B1[Cloud Hosting\textless{}br/\textgreater{}Compute Resources]}
\NormalTok{    C {-}{-}\textgreater{} C1[Model Access\textless{}br/\textgreater{}Analytics Tools]}
\NormalTok{    D {-}{-}\textgreater{} D1[Integrations\textless{}br/\textgreater{}Ecosystem Extensions]}
\NormalTok{    E {-}{-}\textgreater{} E1[Research Projects\textless{}br/\textgreater{}Pilots]}
\NormalTok{    F {-}{-}\textgreater{} F1[Distribution\textless{}br/\textgreater{}Local Presence]}
\end{Highlighting}
\end{Shaded}

\section{6.9 Profitability and Break-Even
Considerations}\label{profitability-and-break-even-considerations}

Prompts:\\
- What is the gross margin target?\\
- At what revenue level does the business break even?\\
- What assumptions influence profitability?\\
- What is the timeline to reach break-even?

\begin{Shaded}
\begin{Highlighting}[]
\NormalTok{flowchart TD}
\NormalTok{    A[Profitability Model] {-}{-}\textgreater{} B[Revenue Growth]}
\NormalTok{    A {-}{-}\textgreater{} C[Cost Structure]}
\NormalTok{    A {-}{-}\textgreater{} D[Gross Margin]}
\NormalTok{    A {-}{-}\textgreater{} E[Break{-}Even Point]}

\NormalTok{    B {-}{-}\textgreater{} B1[Subscriptions\textless{}br/\textgreater{}Services\textless{}br/\textgreater{}Support]}
\NormalTok{    C {-}{-}\textgreater{} C1[Fixed Costs\textless{}br/\textgreater{}Variable Costs]}
\NormalTok{    D {-}{-}\textgreater{} D1[Margin Improvement\textless{}br/\textgreater{}via Automation]}
\NormalTok{    E {-}{-}\textgreater{} E1[Required Revenue\textless{}br/\textgreater{}to Cover Costs]}
\end{Highlighting}
\end{Shaded}

\section{6.10 Strategic Advantages}\label{strategic-advantages}

Summarise what makes the business model defensible and competitive.

Examples: - Integrated ecosystem and workflows\\
- High switching costs\\
- Specialised expertise\\
- Long-term institutional relationships\\
- Automation-driven operating leverage

\begin{Shaded}
\begin{Highlighting}[]
\NormalTok{flowchart LR}
\NormalTok{    A[Strategic Advantages] {-}{-}\textgreater{} B[Integrated Ecosystem]}
\NormalTok{    A {-}{-}\textgreater{} C[High Switching Costs]}
\NormalTok{    A {-}{-}\textgreater{} D[Specialised Expertise]}
\NormalTok{    A {-}{-}\textgreater{} E[Long{-}Term Relationships]}
\NormalTok{    A {-}{-}\textgreater{} F[Automation Leverage]}

\NormalTok{    B {-}{-}\textgreater{} B1[Unified Workflow\textless{}br/\textgreater{}Interoperable Tools]}
\NormalTok{    C {-}{-}\textgreater{} C1[Deep Embedding\textless{}br/\textgreater{}in Customer Processes]}
\NormalTok{    D {-}{-}\textgreater{} D1[Domain Knowledge\textless{}br/\textgreater{}and Proven Methods]}
\NormalTok{    E {-}{-}\textgreater{} E1[Institutional Trust\textless{}br/\textgreater{}Repeat Engagements]}
\NormalTok{    F {-}{-}\textgreater{} F1[Lower Delivery Cost\textless{}br/\textgreater{}Higher Margins]}
\end{Highlighting}
\end{Shaded}

\chapter{7. Product and Service
Portfolio}\label{product-and-service-portfolio}

\section{7.1 Overview}\label{overview}

Describe the complete set of products and services offered.\\
Prompts to complete:\\
- What is included in the portfolio?\\
- How do the offerings relate to each other?\\
- Which solutions are core, which are optional, and which are premium?\\
- How does each offering contribute to the overall business model?

\begin{Shaded}
\begin{Highlighting}[]
\NormalTok{flowchart TD}
\NormalTok{    A[Portfolio Overview] {-}{-}\textgreater{} B[Core Products]}
\NormalTok{    A {-}{-}\textgreater{} C[Add{-}on Modules]}
\NormalTok{    A {-}{-}\textgreater{} D[Professional Services]}
\NormalTok{    A {-}{-}\textgreater{} E[Training \& Support]}
\NormalTok{    A {-}{-}\textgreater{} F[Enterprise Solutions]}

\NormalTok{    B {-}{-}\textgreater{} B1[Main Platforms\textless{}br/\textgreater{}Primary Tools]}
\NormalTok{    C {-}{-}\textgreater{} C1[Optional Enhancements\textless{}br/\textgreater{}Integrations]}
\NormalTok{    D {-}{-}\textgreater{} D1[Implementation\textless{}br/\textgreater{}Consulting]}
\NormalTok{    E {-}{-}\textgreater{} E1[Workshops\textless{}br/\textgreater{}Documentation\textless{}br/\textgreater{}Helpdesk]}
\NormalTok{    F {-}{-}\textgreater{} F1[Large{-}scale Deployments\textless{}br/\textgreater{}Custom Compliance]}
\end{Highlighting}
\end{Shaded}

\section{7.2 Product Categories}\label{product-categories}

Organise offerings into logical groups.

Example structure: - Core Products\\
- Add-on Modules\\
- Professional Services\\
- Training and Support\\
- Enterprise Solutions

\begin{Shaded}
\begin{Highlighting}[]
\NormalTok{flowchart LR}
\NormalTok{    A[Product Categories] {-}{-}\textgreater{} B[Core Products]}
\NormalTok{    A {-}{-}\textgreater{} C[Add{-}on Modules]}
\NormalTok{    A {-}{-}\textgreater{} D[Professional Services]}
\NormalTok{    A {-}{-}\textgreater{} E[Training \& Support]}
\NormalTok{    A {-}{-}\textgreater{} F[Enterprise Solutions]}

\NormalTok{    B {-}{-}\textgreater{} B1[Primary Tools\textless{}br/\textgreater{}Essential Workflows]}
\NormalTok{    C {-}{-}\textgreater{} C1[Optional Features\textless{}br/\textgreater{}Extended Capabilities]}
\NormalTok{    D {-}{-}\textgreater{} D1[Implementation\textless{}br/\textgreater{}Custom Projects]}
\NormalTok{    E {-}{-}\textgreater{} E1[Sessions\textless{}br/\textgreater{}Guides\textless{}br/\textgreater{}Helpdesk]}
\NormalTok{    F {-}{-}\textgreater{} F1[Custom Architecture\textless{}br/\textgreater{}Large Deployments]}
\end{Highlighting}
\end{Shaded}

\section{7.3 Core Products}\label{core-products}

List the primary products that define the value proposition.

For each product include:\\
- What it does\\
- Who it is for\\
- What problems it solves\\
- What outcomes it delivers

Example table:

\begin{longtable}[]{@{}llll@{}}
\toprule\noalign{}
Product & Description & Target User & Key Outcome \\
\midrule\noalign{}
\endhead
\bottomrule\noalign{}
\endlastfoot
Product A & & & \\
Product B & & & \\
Product C & & & \\
\end{longtable}

\begin{Shaded}
\begin{Highlighting}[]
\NormalTok{flowchart TD}
\NormalTok{    A[Core Products] {-}{-}\textgreater{} B[Product A]}
\NormalTok{    A {-}{-}\textgreater{} C[Product B]}
\NormalTok{    A {-}{-}\textgreater{} D[Product C]}

\NormalTok{    B {-}{-}\textgreater{} B1[Description\textless{}br/\textgreater{}Target User\textless{}br/\textgreater{}Outcome]}
\NormalTok{    C {-}{-}\textgreater{} C1[Description\textless{}br/\textgreater{}Target User\textless{}br/\textgreater{}Outcome]}
\NormalTok{    D {-}{-}\textgreater{} D1[Description\textless{}br/\textgreater{}Target User\textless{}br/\textgreater{}Outcome]}
\end{Highlighting}
\end{Shaded}

\section{7.4 Add-on Modules}\label{add-on-modules}

List optional extensions that enrich the core offering.

Prompts:\\
- What additional capabilities can customers purchase?\\
- How do these modules integrate with the core product?\\
- What customer needs do they address?

Example:

\begin{longtable}[]{@{}llll@{}}
\toprule\noalign{}
Module & Purpose & Requirements & Value \\
\midrule\noalign{}
\endhead
\bottomrule\noalign{}
\endlastfoot
Module X & & & \\
Module Y & & & \\
\end{longtable}

\begin{Shaded}
\begin{Highlighting}[]
\NormalTok{flowchart TD}
\NormalTok{    A[Add{-}on Modules] {-}{-}\textgreater{} B[Module X]}
\NormalTok{    A {-}{-}\textgreater{} C[Module Y]}
\NormalTok{    A {-}{-}\textgreater{} D[Module Z]}

\NormalTok{    B {-}{-}\textgreater{} B1[Purpose\textless{}br/\textgreater{}Requirements\textless{}br/\textgreater{}Value]}
\NormalTok{    C {-}{-}\textgreater{} C1[Purpose\textless{}br/\textgreater{}Requirements\textless{}br/\textgreater{}Value]}
\NormalTok{    D {-}{-}\textgreater{} D1[Purpose\textless{}br/\textgreater{}Requirements\textless{}br/\textgreater{}Value]}
\end{Highlighting}
\end{Shaded}

\section{7.5 Professional Services}\label{professional-services}

Describe services that support implementation, optimisation, or
onboarding.

Examples: - Implementation packages\\
- Onboarding programs\\
- Workflow automation consulting\\
- Custom integration projects\\
- Data preparation and modelling support

Include:

\begin{longtable}[]{@{}llll@{}}
\toprule\noalign{}
Service & Description & Deliverables & Duration \\
\midrule\noalign{}
\endhead
\bottomrule\noalign{}
\endlastfoot
Service A & & & \\
Service B & & & \\
\end{longtable}

\begin{Shaded}
\begin{Highlighting}[]
\NormalTok{flowchart LR}
\NormalTok{    A[Professional Services] {-}{-}\textgreater{} B[Service A]}
\NormalTok{    A {-}{-}\textgreater{} C[Service B]}
\NormalTok{    A {-}{-}\textgreater{} D[Service C]}

\NormalTok{    B {-}{-}\textgreater{} B1[Description\textless{}br/\textgreater{}Deliverables\textless{}br/\textgreater{}Duration]}
\NormalTok{    C {-}{-}\textgreater{} C1[Description\textless{}br/\textgreater{}Deliverables\textless{}br/\textgreater{}Duration]}
\NormalTok{    D {-}{-}\textgreater{} D1[Description\textless{}br/\textgreater{}Deliverables\textless{}br/\textgreater{}Duration]}
\end{Highlighting}
\end{Shaded}

\section{7.6 Training and Support}\label{training-and-support}

Outline the training and support offerings.

Prompts:\\
- What training formats are available (live, on-demand,
documentation?)\\
- What support tiers are offered?\\
- What SLAs are provided?

Example:

\begin{longtable}[]{@{}llll@{}}
\toprule\noalign{}
Tier & Support Level & Response Time & Inclusions \\
\midrule\noalign{}
\endhead
\bottomrule\noalign{}
\endlastfoot
Standard & Email & 48 hours & Basic support \\
Pro & Priority & 12--24 hours & Advanced help \\
Enterprise & Dedicated & \textless{} 4 hours & Full support \\
\end{longtable}

\begin{Shaded}
\begin{Highlighting}[]
\NormalTok{flowchart TD}
\NormalTok{    A[Training \& Support] {-}{-}\textgreater{} B[Standard Tier]}
\NormalTok{    A {-}{-}\textgreater{} C[Pro Tier]}
\NormalTok{    A {-}{-}\textgreater{} D[Enterprise Tier]}

\NormalTok{    B {-}{-}\textgreater{} B1[Email Support\textless{}br/\textgreater{}48h Response]}
\NormalTok{    C {-}{-}\textgreater{} C1[Priority Support\textless{}br/\textgreater{}12–24h Response]}
\NormalTok{    D {-}{-}\textgreater{} D1[Dedicated Support\textless{}br/\textgreater{}\&lt; 4h Response]}
\end{Highlighting}
\end{Shaded}

\section{7.7 Enterprise Solutions}\label{enterprise-solutions}

Describe any large-scale deployments or offerings tailored for
institutions.

Prompts:\\
- What customisation options exist?\\
- What architectural or compliance requirements are supported?\\
- What value does this deliver compared to standard packages?

\begin{Shaded}
\begin{Highlighting}[]
\NormalTok{flowchart LR}
\NormalTok{    A[Enterprise Solutions] {-}{-}\textgreater{} B[Custom Architecture]}
\NormalTok{    A {-}{-}\textgreater{} C[Compliance Requirements]}
\NormalTok{    A {-}{-}\textgreater{} D[Large{-}Scale Deployment]}

\NormalTok{    B {-}{-}\textgreater{} B1[Advanced Integrations\textless{}br/\textgreater{}Custom Modules]}
\NormalTok{    C {-}{-}\textgreater{} C1[Data Policies\textless{}br/\textgreater{}Security Controls]}
\NormalTok{    D {-}{-}\textgreater{} D1[High Availability\textless{}br/\textgreater{}Scalability Options]}
\end{Highlighting}
\end{Shaded}

\section{7.8 Product Lifecycle}\label{product-lifecycle}

Outline how each product evolves over time.

Examples of lifecycle phases: - Concept\\
- Pilot\\
- Release\\
- Growth\\
- Maturity\\
- End-of-life

Table:

\begin{longtable}[]{@{}llll@{}}
\toprule\noalign{}
Product & Current Phase & Next Step & Timeline \\
\midrule\noalign{}
\endhead
\bottomrule\noalign{}
\endlastfoot
Product A & & & \\
Product B & & & \\
\end{longtable}

\begin{Shaded}
\begin{Highlighting}[]
\NormalTok{flowchart LR}
\NormalTok{    A[Product Lifecycle] {-}{-}\textgreater{} B[Concept]}
\NormalTok{    B {-}{-}\textgreater{} C[Pilot]}
\NormalTok{    C {-}{-}\textgreater{} D[Release]}
\NormalTok{    D {-}{-}\textgreater{} E[Growth]}
\NormalTok{    E {-}{-}\textgreater{} F[Maturity]}
\NormalTok{    F {-}{-}\textgreater{} G[End{-}of{-}Life]}

\NormalTok{    B {-}{-}\textgreater{} B1[Idea Validation]}
\NormalTok{    C {-}{-}\textgreater{} C1[User Feedback]}
\NormalTok{    D {-}{-}\textgreater{} D1[Market Launch]}
\NormalTok{    E {-}{-}\textgreater{} E1[Feature Expansion]}
\NormalTok{    F {-}{-}\textgreater{} F1[Performance Optimization]}
\NormalTok{    G {-}{-}\textgreater{} G1[Sunset \& Migration]}
\end{Highlighting}
\end{Shaded}

\section{7.9 Portfolio Strategy}\label{portfolio-strategy}

Define how the portfolio fits into the company's long-term strategy.

Prompts:\\
- Which products will drive growth?\\
- Which are loss leaders but necessary?\\
- How do offerings complement each other?\\
- What dependencies or synergies exist?

\begin{Shaded}
\begin{Highlighting}[]
\NormalTok{flowchart TD}
\NormalTok{    A[Portfolio Strategy] {-}{-}\textgreater{} B[Growth Drivers]}
\NormalTok{    A {-}{-}\textgreater{} C[Supporting Offerings]}
\NormalTok{    A {-}{-}\textgreater{} D[Synergies \& Dependencies]}

\NormalTok{    B {-}{-}\textgreater{} B1[Core Products\textless{}br/\textgreater{}Primary Revenue]}
\NormalTok{    C {-}{-}\textgreater{} C1[Add{-}ons\textless{}br/\textgreater{}Services\textless{}br/\textgreater{}Training]}
\NormalTok{    D {-}{-}\textgreater{} D1[Shared Infrastructure\textless{}br/\textgreater{}Common Workflows]}
\end{Highlighting}
\end{Shaded}

\section{7.10 Differentiation Across the
Portfolio}\label{differentiation-across-the-portfolio}

Explain what makes the portfolio unique in the market.

Examples: - Integration between tools\\
- Unified workflow design\\
- Consistent UX\\
- Reproducibility and automation principles\\
- High interoperability\\
- Configurable but standardised solutions

\begin{Shaded}
\begin{Highlighting}[]
\NormalTok{flowchart LR}
\NormalTok{    A[Portfolio Differentiation] {-}{-}\textgreater{} B[Integration]}
\NormalTok{    A {-}{-}\textgreater{} C[Unified UX]}
\NormalTok{    A {-}{-}\textgreater{} D[Automation Principles]}
\NormalTok{    A {-}{-}\textgreater{} E[Interoperability]}
\NormalTok{    A {-}{-}\textgreater{} F[Reproducibility]}

\NormalTok{    B {-}{-}\textgreater{} B1[Connected Tools\textless{}br/\textgreater{}Consistent Data Flow]}
\NormalTok{    C {-}{-}\textgreater{} C1[Common Design\textless{}br/\textgreater{}Shared Patterns]}
\NormalTok{    D {-}{-}\textgreater{} D1[Reduced Manual Steps\textless{}br/\textgreater{}Intelligent Pipelines]}
\NormalTok{    E {-}{-}\textgreater{} E1[Works with Existing Systems\textless{}br/\textgreater{}Low Friction]}
\NormalTok{    F {-}{-}\textgreater{} F1[Repeatable Outputs\textless{}br/\textgreater{}Auditability]}
\end{Highlighting}
\end{Shaded}

\chapter{8. Customer Segments}\label{customer-segments}

\section{8.1 Overview}\label{overview-1}

Define the specific groups of customers the business targets.\\
Prompts to complete:\\
- Who are the primary users and buyers?\\
- What differentiates each segment?\\
- What value does each segment derive?\\
- How large or strategic is each segment?

\begin{Shaded}
\begin{Highlighting}[]
\NormalTok{flowchart TD}
\NormalTok{    A[Customer Segments] {-}{-}\textgreater{} B[Primary Segments]}
\NormalTok{    A {-}{-}\textgreater{} C[Secondary Segments]}

\NormalTok{    B {-}{-}\textgreater{} B1[Segment A]}
\NormalTok{    B {-}{-}\textgreater{} B2[Segment B]}
\NormalTok{    B {-}{-}\textgreater{} B3[Segment C]}

\NormalTok{    C {-}{-}\textgreater{} C1[Segment X]}
\NormalTok{    C {-}{-}\textgreater{} C2[Segment Y]}
\end{Highlighting}
\end{Shaded}

\section{8.2 Segment Definitions}\label{segment-definitions}

\subsection{8.2.1 Primary Segments}\label{primary-segments}

List your main customer types.

\begin{longtable}[]{@{}
  >{\raggedright\arraybackslash}p{(\linewidth - 6\tabcolsep) * \real{0.1286}}
  >{\raggedright\arraybackslash}p{(\linewidth - 6\tabcolsep) * \real{0.1857}}
  >{\raggedright\arraybackslash}p{(\linewidth - 6\tabcolsep) * \real{0.3571}}
  >{\raggedright\arraybackslash}p{(\linewidth - 6\tabcolsep) * \real{0.3286}}@{}}
\toprule\noalign{}
\begin{minipage}[b]{\linewidth}\raggedright
Segment
\end{minipage} & \begin{minipage}[b]{\linewidth}\raggedright
Description
\end{minipage} & \begin{minipage}[b]{\linewidth}\raggedright
Role in Buying Process
\end{minipage} & \begin{minipage}[b]{\linewidth}\raggedright
Strategic Importance
\end{minipage} \\
\midrule\noalign{}
\endhead
\bottomrule\noalign{}
\endlastfoot
Segment A & & & \\
Segment B & & & \\
Segment C & & & \\
\end{longtable}

\subsection{8.2.2 Secondary Segments}\label{secondary-segments}

Additional audiences that may adopt or influence adoption.

\begin{longtable}[]{@{}lll@{}}
\toprule\noalign{}
Segment & Description & Influence Level \\
\midrule\noalign{}
\endhead
\bottomrule\noalign{}
\endlastfoot
Segment X & & \\
Segment Y & & \\
\end{longtable}

\begin{Shaded}
\begin{Highlighting}[]
\NormalTok{flowchart LR}
\NormalTok{    A[Segment Definitions] {-}{-}\textgreater{} B[Primary]}
\NormalTok{    A {-}{-}\textgreater{} C[Secondary]}

\NormalTok{    B {-}{-}\textgreater{} B1[Segment A\textless{}br/\textgreater{}Role | Influence | Importance]}
\NormalTok{    B {-}{-}\textgreater{} B2[Segment B\textless{}br/\textgreater{}Role | Influence | Importance]}
\NormalTok{    B {-}{-}\textgreater{} B3[Segment C\textless{}br/\textgreater{}Role | Influence | Importance]}

\NormalTok{    C {-}{-}\textgreater{} C1[Segment X\textless{}br/\textgreater{}Influence Level]}
\NormalTok{    C {-}{-}\textgreater{} C2[Segment Y\textless{}br/\textgreater{}Influence Level]}
\end{Highlighting}
\end{Shaded}

\section{8.3 Customer Needs and
Motivations}\label{customer-needs-and-motivations}

Describe what each segment wants to achieve.

\begin{longtable}[]{@{}llll@{}}
\toprule\noalign{}
Segment & Key Needs & Motivations & Barriers \\
\midrule\noalign{}
\endhead
\bottomrule\noalign{}
\endlastfoot
Segment A & & & \\
Segment B & & & \\
\end{longtable}

Prompts:\\
- What outcomes matter most?\\
- What pressures or constraints do they have?\\
- What would make them switch from a current solution?

\begin{Shaded}
\begin{Highlighting}[]
\NormalTok{flowchart TD}
\NormalTok{    A[Needs \& Motivations] {-}{-}\textgreater{} B[Segment A]}
\NormalTok{    A {-}{-}\textgreater{} C[Segment B]}

\NormalTok{    B {-}{-}\textgreater{} B1[Key Needs]}
\NormalTok{    B {-}{-}\textgreater{} B2[Motivations]}
\NormalTok{    B {-}{-}\textgreater{} B3[Barriers]}

\NormalTok{    C {-}{-}\textgreater{} C1[Key Needs]}
\NormalTok{    C {-}{-}\textgreater{} C2[Motivations]}
\NormalTok{    C {-}{-}\textgreater{} C3[Barriers]}
\end{Highlighting}
\end{Shaded}

\section{8.4 Buying Criteria}\label{buying-criteria}

Explain what factors drive purchasing decisions.

Examples: - Price sensitivity\\
- Ease of implementation\\
- Integration compatibility\\
- Automation level\\
- Support and training availability\\
- Compliance requirements\\
- Proven reliability

Table format:

\begin{longtable}[]{@{}lll@{}}
\toprule\noalign{}
Criterion & Importance & Notes \\
\midrule\noalign{}
\endhead
\bottomrule\noalign{}
\endlastfoot
Price & & \\
Ease of Use & & \\
Automation & & \\
Integration & & \\
\end{longtable}

\begin{Shaded}
\begin{Highlighting}[]
\NormalTok{flowchart LR}
\NormalTok{    A[Buying Criteria] {-}{-}\textgreater{} B[Price]}
\NormalTok{    A {-}{-}\textgreater{} C[Ease of Use]}
\NormalTok{    A {-}{-}\textgreater{} D[Automation]}
\NormalTok{    A {-}{-}\textgreater{} E[Integration]}
\NormalTok{    A {-}{-}\textgreater{} F[Compliance]}
\NormalTok{    A {-}{-}\textgreater{} G[Reliability]}
\end{Highlighting}
\end{Shaded}

\section{8.5 Segment Prioritisation}\label{segment-prioritisation}

Describe which segments you should target first and why.

Considerations: - Market size\\
- Revenue potential\\
- Speed of adoption\\
- Strategic fit\\
- Competitive landscape

Example table:

\begin{longtable}[]{@{}lll@{}}
\toprule\noalign{}
Segment & Priority Level & Reason \\
\midrule\noalign{}
\endhead
\bottomrule\noalign{}
\endlastfoot
Segment A & High & \\
Segment B & Medium & \\
Segment C & Low & \\
\end{longtable}

\begin{Shaded}
\begin{Highlighting}[]
\NormalTok{flowchart TD}
\NormalTok{    A[Segment Prioritisation] {-}{-}\textgreater{} B[High Priority]}
\NormalTok{    A {-}{-}\textgreater{} C[Medium Priority]}
\NormalTok{    A {-}{-}\textgreater{} D[Low Priority]}

\NormalTok{    B {-}{-}\textgreater{} B1[Segment A]}
\NormalTok{    C {-}{-}\textgreater{} C1[Segment B]}
\NormalTok{    D {-}{-}\textgreater{} D1[Segment C]}
\end{Highlighting}
\end{Shaded}

\section{8.6 Customer Persona
Placeholders}\label{customer-persona-placeholders}

Create templates for detailed personas (filled in separately).

\subsection{Persona Template:}\label{persona-template}

\begin{itemize}
\tightlist
\item
  Background\\
\item
  Role and responsibilities\\
\item
  Goals\\
\item
  Challenges\\
\item
  Buying process\\
\item
  Success indicators\\
\item
  Preferred communication channels
\end{itemize}

(Placeholder for persona cards)

\section{8.7 Customer Journey Overview}\label{customer-journey-overview}

Describe how each segment interacts with the business across stages.

Stages to consider: - Awareness\\
- Consideration\\
- Decision\\
- Onboarding\\
- Usage\\
- Renewal

Example table:

\begin{longtable}[]{@{}llll@{}}
\toprule\noalign{}
Journey Stage & Customer Actions & Business Actions & KPIs \\
\midrule\noalign{}
\endhead
\bottomrule\noalign{}
\endlastfoot
Awareness & & & \\
Decision & & & \\
Renewal & & & \\
\end{longtable}

\begin{Shaded}
\begin{Highlighting}[]
\NormalTok{flowchart LR}
\NormalTok{    A[Awareness] {-}{-}\textgreater{} B[Consideration]}
\NormalTok{    B {-}{-}\textgreater{} C[Decision]}
\NormalTok{    C {-}{-}\textgreater{} D[Onboarding]}
\NormalTok{    D {-}{-}\textgreater{} E[Usage]}
\NormalTok{    E {-}{-}\textgreater{} F[Renewal]}

\NormalTok{    A {-}{-}\textgreater{} A1[Content | Outreach]}
\NormalTok{    B {-}{-}\textgreater{} B1[Demos | Evaluation]}
\NormalTok{    C {-}{-}\textgreater{} C1[Contracting]}
\NormalTok{    D {-}{-}\textgreater{} D1[Setup | Training]}
\NormalTok{    E {-}{-}\textgreater{} E1[Daily Value | Support]}
\NormalTok{    F {-}{-}\textgreater{} F1[Review | Expansion]}
\end{Highlighting}
\end{Shaded}

\section{8.8 Segment Pain Points}\label{segment-pain-points}

Document recurring problems within each segment.

\begin{longtable}[]{@{}llll@{}}
\toprule\noalign{}
Segment & Pain Point & Impact Severity & Opportunity \\
\midrule\noalign{}
\endhead
\bottomrule\noalign{}
\endlastfoot
Segment A & & & \\
Segment B & & & \\
\end{longtable}

\begin{Shaded}
\begin{Highlighting}[]
\NormalTok{flowchart TD}
\NormalTok{    A[Pain Points] {-}{-}\textgreater{} B[Segment A]}
\NormalTok{    A {-}{-}\textgreater{} C[Segment B]}

\NormalTok{    B {-}{-}\textgreater{} B1[High Impact Issues]}
\NormalTok{    C {-}{-}\textgreater{} C1[High Impact Issues]}
\end{Highlighting}
\end{Shaded}

\section{8.9 Value Alignment}\label{value-alignment}

Describe how the offering aligns with each segment's needs.

Prompts:\\
- Which product features directly support which segments?\\
- Where is value strongest?\\
- Which gaps need addressing?

Example table:

\begin{longtable}[]{@{}llll@{}}
\toprule\noalign{}
Segment & Relevant Features & Strength of Fit & Notes \\
\midrule\noalign{}
\endhead
\bottomrule\noalign{}
\endlastfoot
Segment A & & & \\
Segment B & & & \\
\end{longtable}

\begin{Shaded}
\begin{Highlighting}[]
\NormalTok{flowchart LR}
\NormalTok{    A[Value Alignment] {-}{-}\textgreater{} B[Segment A]}
\NormalTok{    A {-}{-}\textgreater{} C[Segment B]}

\NormalTok{    B {-}{-}\textgreater{} B1[Relevant Features\textless{}br/\textgreater{}Strength of Fit]}
\NormalTok{    C {-}{-}\textgreater{} C1[Relevant Features\textless{}br/\textgreater{}Strength of Fit]}
\end{Highlighting}
\end{Shaded}

\chapter{9. Competitive Landscape}\label{competitive-landscape}

\section{9.1 Overview}\label{overview-2}

Describe the competitive environment the business operates in.\\
Prompts to complete:\\
- Who are the direct competitors?\\
- Who are the indirect competitors or substitutes?\\
- What alternatives do customers currently rely on?\\
- What competitive forces shape the market?

(Placeholder for competitive landscape map)

\section{9.2 Direct Competitors}\label{direct-competitors}

Companies offering similar solutions targeting the same customer
segments.

\begin{longtable}[]{@{}llll@{}}
\toprule\noalign{}
Competitor & Description & Strengths & Weaknesses \\
\midrule\noalign{}
\endhead
\bottomrule\noalign{}
\endlastfoot
Competitor A & & & \\
Competitor B & & & \\
Competitor C & & & \\
\end{longtable}

(Placeholder for direct competitor radar chart)

\section{9.3 Indirect Competitors}\label{indirect-competitors}

Tools or services that solve related problems but in different ways.

\begin{longtable}[]{@{}llll@{}}
\toprule\noalign{}
Competitor & Substitute for & Key Advantage & Key Limitation \\
\midrule\noalign{}
\endhead
\bottomrule\noalign{}
\endlastfoot
Alternative A & & & \\
Alternative B & & & \\
\end{longtable}

(Placeholder for substitution matrix)

\section{9.4 Internal or In-House
Alternatives}\label{internal-or-in-house-alternatives}

Describe what potential customers may already be using internally.

Examples: - Custom scripts\\
- Spreadsheets\\
- Manual workflows\\
- Legacy systems\\
- Fragmented toolchains

\begin{longtable}[]{@{}lll@{}}
\toprule\noalign{}
Internal Solution & Pain Points & Why Customers Switch \\
\midrule\noalign{}
\endhead
\bottomrule\noalign{}
\endlastfoot
Approach A & & \\
Approach B & & \\
\end{longtable}

(Placeholder for build-vs-buy comparison)

\section{9.5 Competitive Factors (Porter's
Lens)}\label{competitive-factors-porters-lens}

Identify factors that influence competition.

Key factors may include: - Ease of switching\\
- Cost sensitivity\\
- Market fragmentation\\
- Threat of substitutes\\
- Supplier dependency\\
- New entrant feasibility

(Placeholder for five forces diagram)

\section{9.6 Feature Comparison}\label{feature-comparison}

Compare your solution against competitors across critical dimensions.

\begin{longtable}[]{@{}llll@{}}
\toprule\noalign{}
Feature & Your Solution & Competitor A & Competitor B \\
\midrule\noalign{}
\endhead
\bottomrule\noalign{}
\endlastfoot
Automation & & & \\
AI Capabilities & & & \\
Integration Support & & & \\
Security/Compliance & & & \\
Ease of Use & & & \\
Cost & & & \\
\end{longtable}

(Placeholder for scoring framework)

\section{9.7 Market Positioning}\label{market-positioning}

Describe how the business positions itself relative to competitors.

Prompts:\\
- What axis defines differentiation (price vs performance, automation vs
manual, etc)?\\
- Where do competitors cluster?\\
- What unique space does your solution occupy?

(Placeholder for perceptual positioning map)

\section{9.8 Competitive Advantages}\label{competitive-advantages}

List your defensible strengths.

Examples: - Integrated architecture across multiple workflows\\
- Automation-first approach\\
- Lower implementation friction\\
- Strong domain expertise\\
- Interoperability with existing systems\\
- Insight and reporting capabilities\\
- Reproducibility and auditability

(Placeholder for competitive moat diagram)

\section{9.9 Market Gaps and
Opportunities}\label{market-gaps-and-opportunities}

Identify areas where competitors are weak or absent.

\begin{longtable}[]{@{}llll@{}}
\toprule\noalign{}
Opportunity Area & Gap in Market & Your Advantage & Priority \\
\midrule\noalign{}
\endhead
\bottomrule\noalign{}
\endlastfoot
Example A & & & \\
Example B & & & \\
\end{longtable}

(Placeholder for opportunity landscape)

\section{9.10 Risks From Competitors}\label{risks-from-competitors}

Document potential threats.

\begin{longtable}[]{@{}llll@{}}
\toprule\noalign{}
Competitor Action & Risk Level & Impact & Mitigation \\
\midrule\noalign{}
\endhead
\bottomrule\noalign{}
\endlastfoot
Price pressure & & & \\
Feature replication & & & \\
Aggressive bundling & & & \\
\end{longtable}

(Placeholder for risk matrix)

\section{9.11 Strategic Response}\label{strategic-response}

Describe how the business will maintain a competitive edge.

Prompts:\\
- What investments reinforce your strongest advantages?\\
- What partnerships create defensibility?\\
- What innovations must remain ahead of competitors?

(Placeholder for strategic action plan)

\chapter{10. Go To Market Strategy}\label{go-to-market-strategy}

\section{10.1 Overview}\label{overview-3}

Define how the business brings its products and services to customers.\\
Prompts to complete:\\
- What is the overall GTM approach?\\
- Which segments are targeted first and why?\\
- What channels will drive awareness, acquisition, and conversion?

(Placeholder for GTM framework)

\section{10.2 Target Segments for GTM}\label{target-segments-for-gtm}

Identify the priority audiences for launch and early scaling.

\begin{longtable}[]{@{}
  >{\raggedright\arraybackslash}p{(\linewidth - 6\tabcolsep) * \real{0.1268}}
  >{\raggedright\arraybackslash}p{(\linewidth - 6\tabcolsep) * \real{0.2676}}
  >{\raggedright\arraybackslash}p{(\linewidth - 6\tabcolsep) * \real{0.2958}}
  >{\raggedright\arraybackslash}p{(\linewidth - 6\tabcolsep) * \real{0.3099}}@{}}
\toprule\noalign{}
\begin{minipage}[b]{\linewidth}\raggedright
Segment
\end{minipage} & \begin{minipage}[b]{\linewidth}\raggedright
Why Target First
\end{minipage} & \begin{minipage}[b]{\linewidth}\raggedright
Expected Conversion
\end{minipage} & \begin{minipage}[b]{\linewidth}\raggedright
Strategic Importance
\end{minipage} \\
\midrule\noalign{}
\endhead
\bottomrule\noalign{}
\endlastfoot
Segment A & & & \\
Segment B & & & \\
Segment C & & & \\
\end{longtable}

(Placeholder for segment prioritisation visual)

\section{10.3 Value Messaging}\label{value-messaging}

Define the core messages communicated to each segment.

\begin{longtable}[]{@{}llll@{}}
\toprule\noalign{}
Segment & Key Message & Pain Addressed & Desired Reaction \\
\midrule\noalign{}
\endhead
\bottomrule\noalign{}
\endlastfoot
Segment A & & & \\
Segment B & & & \\
\end{longtable}

Prompts:\\
- What transformation is being promised?\\
- What emotional or functional value is emphasised?\\
- What evidence or proof points support the message?

(Placeholder for messaging architecture)

\section{10.4 Acquisition Channels}\label{acquisition-channels}

List the channels used to attract potential customers.

Examples: - LinkedIn organic presence\\
- Direct outreach to institutions\\
- Webinars and workshops\\
- Industry conferences\\
- Partnerships with academic or research networks\\
- Referral programs\\
- SEO and content marketing

Table format:

\begin{longtable}[]{@{}
  >{\raggedright\arraybackslash}p{(\linewidth - 6\tabcolsep) * \real{0.2195}}
  >{\raggedright\arraybackslash}p{(\linewidth - 6\tabcolsep) * \real{0.2439}}
  >{\raggedright\arraybackslash}p{(\linewidth - 6\tabcolsep) * \real{0.2683}}
  >{\raggedright\arraybackslash}p{(\linewidth - 6\tabcolsep) * \real{0.2683}}@{}}
\toprule\noalign{}
\begin{minipage}[b]{\linewidth}\raggedright
Channel
\end{minipage} & \begin{minipage}[b]{\linewidth}\raggedright
Purpose
\end{minipage} & \begin{minipage}[b]{\linewidth}\raggedright
Strength
\end{minipage} & \begin{minipage}[b]{\linewidth}\raggedright
Challenge
\end{minipage} \\
\midrule\noalign{}
\endhead
\bottomrule\noalign{}
\endlastfoot
LinkedIn & Awareness & Targeted reach & Competes for attention \\
Partnerships & Trust building & High credibility & Long negotiation
cycles \\
\end{longtable}

(Placeholder for multi-channel funnel diagram)

\section{10.5 Lead Generation Strategy}\label{lead-generation-strategy}

Describe activities used to produce qualified leads.

Prompts:\\
- What content attracts your target users?\\
- What incentives encourage sign-ups or demos?\\
- What KPIs measure lead quality?

Example structure: - Educational posts and insights\\
- Live demos\\
- Whitepapers and case studies\\
- Free diagnostic tools\\
- AI-powered examples or previews

(Placeholder for lead magnet map)

\section{10.6 Sales Process}\label{sales-process}

Define each stage of the sales cycle.

\begin{longtable}[]{@{}
  >{\raggedright\arraybackslash}p{(\linewidth - 6\tabcolsep) * \real{0.2286}}
  >{\raggedright\arraybackslash}p{(\linewidth - 6\tabcolsep) * \real{0.3714}}
  >{\raggedright\arraybackslash}p{(\linewidth - 6\tabcolsep) * \real{0.2286}}
  >{\raggedright\arraybackslash}p{(\linewidth - 6\tabcolsep) * \real{0.1714}}@{}}
\toprule\noalign{}
\begin{minipage}[b]{\linewidth}\raggedright
Stage
\end{minipage} & \begin{minipage}[b]{\linewidth}\raggedright
Description
\end{minipage} & \begin{minipage}[b]{\linewidth}\raggedright
Owner
\end{minipage} & \begin{minipage}[b]{\linewidth}\raggedright
KPI
\end{minipage} \\
\midrule\noalign{}
\endhead
\bottomrule\noalign{}
\endlastfoot
Discovery & Initial call or meeting & Sales / Founder & Qualification
rate \\
Demo & Product walkthrough & Product / Founder & Demo-to-proposal
ratio \\
Proposal & Scope and pricing & Sales / Consulting & Acceptance rate \\
Closing & Contracting and negotiation & Leadership & Close rate \\
Onboarding & Implementation kickoff & Delivery & Activation \% \\
\end{longtable}

(Placeholder for sales pipeline diagram)

\section{10.7 Partnership Strategy}\label{partnership-strategy}

Explain how partnerships will support customer acquisition and
expansion.

Examples: - Academic alliances\\
- Technology partnerships\\
- Joint research projects\\
- Co-branded workshops\\
- Integration partners

Prompts:\\
- What partners increase reach or credibility?\\
- What incentives exist for partners?\\
- What is the partnership onboarding process?

(Placeholder for partner ecosystem map)

\section{10.8 Pricing and Packaging Strategy for
GTM}\label{pricing-and-packaging-strategy-for-gtm}

Describe how pricing aligns with GTM positioning.

Prompts:\\
- Is entry pricing designed to lower friction?\\
- Are packages aligned with segment maturity?\\
- Are there upsell paths?

Example:

\begin{longtable}[]{@{}llll@{}}
\toprule\noalign{}
Package & Target Segment & Purpose & Upsell Path \\
\midrule\noalign{}
\endhead
\bottomrule\noalign{}
\endlastfoot
Starter & Small teams & Lower barrier to entry & Pro plan \\
Pro & Mid-size orgs & Core value delivery & Enterprise \\
Enterprise & Institutions & Maximum flexibility & Add-ons \\
\end{longtable}

(Placeholder for packaging map)

\section{10.9 Customer Onboarding
Strategy}\label{customer-onboarding-strategy}

Outline how new customers are activated and guided toward value.

Prompts:\\
- What does onboarding include?\\
- How long does it take to reach first value?\\
- What training or materials are required?

Example:

\begin{longtable}[]{@{}
  >{\raggedright\arraybackslash}p{(\linewidth - 6\tabcolsep) * \real{0.1667}}
  >{\raggedright\arraybackslash}p{(\linewidth - 6\tabcolsep) * \real{0.3056}}
  >{\raggedright\arraybackslash}p{(\linewidth - 6\tabcolsep) * \real{0.2222}}
  >{\raggedright\arraybackslash}p{(\linewidth - 6\tabcolsep) * \real{0.3056}}@{}}
\toprule\noalign{}
\begin{minipage}[b]{\linewidth}\raggedright
Step
\end{minipage} & \begin{minipage}[b]{\linewidth}\raggedright
Activity
\end{minipage} & \begin{minipage}[b]{\linewidth}\raggedright
Owner
\end{minipage} & \begin{minipage}[b]{\linewidth}\raggedright
Objective
\end{minipage} \\
\midrule\noalign{}
\endhead
\bottomrule\noalign{}
\endlastfoot
Kickoff & Alignment meeting & Delivery & Expectations set \\
Setup & Configuration & Engineering & System operational \\
Training & Live or guided training & Education & User competence \\
First Output & Running first workflow/report & Customer & Value
realised \\
\end{longtable}

(Placeholder for onboarding flow)

\section{10.10 Expansion and Retention
Strategy}\label{expansion-and-retention-strategy}

Describe how existing customers grow and remain engaged.

Prompts:\\
- What signals indicate upsell opportunities?\\
- What feedback loops improve retention?\\
- How are success stories shared?

Example retention levers: - Continuous updates\\
- Dedicated support\\
- Training refreshers\\
- Community or knowledge hubs\\
- Quarterly business reviews

(Placeholder for customer lifecycle diagram)

\section{10.11 KPIs for GTM Success}\label{kpis-for-gtm-success}

Define measurable indicators that validate GTM effectiveness.

\begin{longtable}[]{@{}lll@{}}
\toprule\noalign{}
KPI Category & KPI & Target \\
\midrule\noalign{}
\endhead
\bottomrule\noalign{}
\endlastfoot
Acquisition & Lead volume & \\
Conversion & Demo-to-close rate & \\
Revenue & Average contract value & \\
Retention & Renewal rate & \\
Awareness & Engagement metrics & \\
\end{longtable}

(Placeholder for KPI dashboard)

\section{10.12 Risks and Mitigation in
GTM}\label{risks-and-mitigation-in-gtm}

Identify threats that may slow GTM performance.

\begin{longtable}[]{@{}llll@{}}
\toprule\noalign{}
Risk & Impact & Likelihood & Mitigation \\
\midrule\noalign{}
\endhead
\bottomrule\noalign{}
\endlastfoot
Low lead quality & & & \\
Message-market mismatch & & & \\
Channel saturation & & & \\
\end{longtable}

(Placeholder for GTM risk matrix)

\section{10.13 GTM Timeline}\label{gtm-timeline}

Provide a phased plan for rollout.

\begin{longtable}[]{@{}llll@{}}
\toprule\noalign{}
Phase & Timeframe & Activities & Expected Outcomes \\
\midrule\noalign{}
\endhead
\bottomrule\noalign{}
\endlastfoot
Phase 1 & 0--3 months & & \\
Phase 2 & 3--6 months & & \\
Phase 3 & 6--12 months & & \\
\end{longtable}

(Placeholder for timeline graphic)

\chapter{11. Operations Plan}\label{operations-plan}

\section{11.1 Overview}\label{overview-4}

Describe how the organisation delivers its products and services on a
daily basis.

Prompts to complete:\\
- What are the core operational activities?\\
- How does the business ensure reliability, quality, and consistency?\\
- What systems and processes are required to support operations?

(Placeholder for operational architecture)

\section{11.2 Operating Model}\label{operating-model}

Define how the organisation functions internally.

Components to consider: - Delivery model (remote, hybrid, onsite)\\
- Team structure and responsibilities\\
- Interaction between product, engineering, sales, and support\\
- Escalation paths and decision-making structure

Example table:

\begin{longtable}[]{@{}lll@{}}
\toprule\noalign{}
Function & Responsibility & Key Outputs \\
\midrule\noalign{}
\endhead
\bottomrule\noalign{}
\endlastfoot
Delivery & Implementations & Configured systems \\
Support & Customer inquiries & Issue resolution \\
Engineering & Product updates & Feature releases \\
Operations & Admin + compliance & Smooth day-to-day running \\
\end{longtable}

(Placeholder for operating model diagram)

\section{11.3 Core Processes}\label{core-processes}

List and describe the main repeatable processes that ensure operational
excellence.

Examples: - Customer onboarding\\
- Project delivery workflow\\
- Incident and issue management\\
- Release management\\
- Quality assurance\\
- Documentation and knowledge management

(Placeholder for process flow maps)

\section{11.4 Delivery Workflow}\label{delivery-workflow}

Outline the detailed workflow used to deliver services to customers.

Prompts:\\
- What steps occur from contract signing to completed delivery?\\
- What are the internal and external handoffs?\\
- Which tools support each stage?

Example:

\begin{longtable}[]{@{}llll@{}}
\toprule\noalign{}
Step & Description & Owner & Deliverable \\
\midrule\noalign{}
\endhead
\bottomrule\noalign{}
\endlastfoot
Kickoff & Scope + alignment & Delivery & Project plan \\
Configuration & Technical setup & Engineering & Live environment \\
Validation & Testing + feedback & Customer & Acceptance \\
Handover & Documentation + training & Delivery & Final sign-off \\
\end{longtable}

(Placeholder for service delivery lifecycle)

\section{11.5 Quality Management}\label{quality-management}

Explain how quality is maintained throughout operations.

Prompts:\\
- What standards or guidelines ensure consistent quality?\\
- What review mechanisms exist?\\
- How are customer complaints or defects handled and tracked?

Consider topics: - QA checklists\\
- Internal reviews\\
- Automated testing workflows\\
- Error logging and monitoring\\
- Customer satisfaction surveys

(Placeholder for QA framework)

\section{11.6 Resource Requirements}\label{resource-requirements}

Identify the human and technical resources needed to operate
effectively.

\subsection{Human Resources}\label{human-resources}

\begin{itemize}
\tightlist
\item
  Delivery specialists\\
\item
  Technical support\\
\item
  Engineers\\
\item
  Project managers\\
\item
  Trainers\\
\item
  Operations staff
\end{itemize}

\subsection{Technical Resources}\label{technical-resources}

\begin{itemize}
\tightlist
\item
  Hosting infrastructure\\
\item
  CI/CD pipelines\\
\item
  Monitoring tools\\
\item
  Ticketing systems\\
\item
  Documentation systems
\end{itemize}

(Placeholder for resource allocation model)

\section{11.7 Tools and Systems}\label{tools-and-systems}

List the tools used across the organisation.

Examples: - Project management system\\
- CRM or pipeline tracker\\
- Knowledge base\\
- Support ticketing\\
- Monitoring and alerting\\
- Automation and CI/CD tools

(Placeholder for system landscape map)

\section{11.8 Supplier and Partner
Dependencies}\label{supplier-and-partner-dependencies}

Document any external tools or partners essential for operations.

Examples: - Cloud hosting provider\\
- AI model infrastructure\\
- Payment processors\\
- Integration vendors\\
- Security or compliance partners

(Placeholder for supplier dependency chart)

\section{11.9 Risk and Continuity
Planning}\label{risk-and-continuity-planning}

Outline how the organisation ensures continuity in operations.

Include: - Disaster recovery planning\\
- Data backup policies\\
- Incident response procedures\\
- Redundancy for critical services\\
- Business continuity strategy

(Placeholder for continuity plan diagram)

\section{11.10 Operational KPIs}\label{operational-kpis}

Define the KPIs that measure operational performance.

\begin{longtable}[]{@{}
  >{\raggedright\arraybackslash}p{(\linewidth - 6\tabcolsep) * \real{0.3415}}
  >{\raggedright\arraybackslash}p{(\linewidth - 6\tabcolsep) * \real{0.1463}}
  >{\raggedright\arraybackslash}p{(\linewidth - 6\tabcolsep) * \real{0.3171}}
  >{\raggedright\arraybackslash}p{(\linewidth - 6\tabcolsep) * \real{0.1951}}@{}}
\toprule\noalign{}
\begin{minipage}[b]{\linewidth}\raggedright
KPI Category
\end{minipage} & \begin{minipage}[b]{\linewidth}\raggedright
KPI
\end{minipage} & \begin{minipage}[b]{\linewidth}\raggedright
Definition
\end{minipage} & \begin{minipage}[b]{\linewidth}\raggedright
Target
\end{minipage} \\
\midrule\noalign{}
\endhead
\bottomrule\noalign{}
\endlastfoot
Delivery & On-time completion & \% of projects delivered on schedule
& \\
Quality & Defect rate & Issues reported per project & \\
Support & Response time & Average first reply & \\
Efficiency & Cycle time & Time per workflow & \\
Satisfaction & Customer score & NPS or equivalent & \\
\end{longtable}

(Placeholder for KPI dashboard)

\section{11.11 Scaling Operations}\label{scaling-operations}

Describe how operations evolve as the organisation grows.

Prompts:\\
- What processes must scale first?\\
- What can be automated?\\
- What organisational changes support larger scale?

Example scaling levers: - Improved automation\\
- Standardised delivery packages\\
- Dedicated support function\\
- Additional engineering capacity\\
- Partner-driven implementations

(Placeholder for scaling roadmap)

\chapter{12. Financial Plan}\label{financial-plan}

\section{12.1 Overview}\label{overview-5}

Define the financial structure, projections, and assumptions that
support the business strategy.

Prompts to complete:\\
- What is the revenue model and expected growth path?\\
- What are the major cost drivers?\\
- What assumptions underpin the financial forecasts?\\
- What milestones define financial success?

(Placeholder for financial summary table)

\section{12.2 Revenue Projections}\label{revenue-projections}

Provide multi-year revenue forecasts based on product lines or customer
segments.

Example table:

\begin{longtable}[]{@{}
  >{\raggedright\arraybackslash}p{(\linewidth - 8\tabcolsep) * \real{0.0759}}
  >{\raggedright\arraybackslash}p{(\linewidth - 8\tabcolsep) * \real{0.2405}}
  >{\raggedright\arraybackslash}p{(\linewidth - 8\tabcolsep) * \real{0.2405}}
  >{\raggedright\arraybackslash}p{(\linewidth - 8\tabcolsep) * \real{0.2405}}
  >{\raggedright\arraybackslash}p{(\linewidth - 8\tabcolsep) * \real{0.2025}}@{}}
\toprule\noalign{}
\begin{minipage}[b]{\linewidth}\raggedright
Year
\end{minipage} & \begin{minipage}[b]{\linewidth}\raggedright
Revenue Stream A
\end{minipage} & \begin{minipage}[b]{\linewidth}\raggedright
Revenue Stream B
\end{minipage} & \begin{minipage}[b]{\linewidth}\raggedright
Revenue Stream C
\end{minipage} & \begin{minipage}[b]{\linewidth}\raggedright
Total Revenue
\end{minipage} \\
\midrule\noalign{}
\endhead
\bottomrule\noalign{}
\endlastfoot
Year 1 & & & & \\
Year 2 & & & & \\
Year 3 & & & & \\
\end{longtable}

Prompts:\\
- What penetration rate is assumed for each segment?\\
- What is the expected customer lifetime value?\\
- What is the churn assumption?

(Placeholder for revenue projection chart)

\section{12.3 Cost Projections}\label{cost-projections}

Outline expected operating costs over time.

Categories to include:\\
- Infrastructure and hosting\\
- Product development\\
- Support and delivery\\
- Sales and marketing\\
- Administrative overhead\\
- Legal and compliance

Example table:

\begin{longtable}[]{@{}
  >{\raggedright\arraybackslash}p{(\linewidth - 12\tabcolsep) * \real{0.0706}}
  >{\raggedright\arraybackslash}p{(\linewidth - 12\tabcolsep) * \real{0.1765}}
  >{\raggedright\arraybackslash}p{(\linewidth - 12\tabcolsep) * \real{0.1412}}
  >{\raggedright\arraybackslash}p{(\linewidth - 12\tabcolsep) * \real{0.1059}}
  >{\raggedright\arraybackslash}p{(\linewidth - 12\tabcolsep) * \real{0.2118}}
  >{\raggedright\arraybackslash}p{(\linewidth - 12\tabcolsep) * \real{0.1412}}
  >{\raggedright\arraybackslash}p{(\linewidth - 12\tabcolsep) * \real{0.1529}}@{}}
\toprule\noalign{}
\begin{minipage}[b]{\linewidth}\raggedright
Year
\end{minipage} & \begin{minipage}[b]{\linewidth}\raggedright
Infrastructure
\end{minipage} & \begin{minipage}[b]{\linewidth}\raggedright
Development
\end{minipage} & \begin{minipage}[b]{\linewidth}\raggedright
Support
\end{minipage} & \begin{minipage}[b]{\linewidth}\raggedright
Sales/Marketing
\end{minipage} & \begin{minipage}[b]{\linewidth}\raggedright
Operations
\end{minipage} & \begin{minipage}[b]{\linewidth}\raggedright
Total Costs
\end{minipage} \\
\midrule\noalign{}
\endhead
\bottomrule\noalign{}
\endlastfoot
Year 1 & & & & & & \\
Year 2 & & & & & & \\
Year 3 & & & & & & \\
\end{longtable}

(Placeholder for cost projection chart)

\section{12.4 Gross Margin and Net
Margin}\label{gross-margin-and-net-margin}

Define the margin model and targets.

Prompts:\\
- What is the gross margin expectation for each product?\\
- How do margins evolve as automation increases?\\
- What is the breakeven point?

Example table:

\begin{longtable}[]{@{}llll@{}}
\toprule\noalign{}
Metric & Year 1 & Year 2 & Year 3 \\
\midrule\noalign{}
\endhead
\bottomrule\noalign{}
\endlastfoot
Gross Margin & & & \\
Net Margin & & & \\
\end{longtable}

(Placeholder for margin waterfall chart)

\section{12.5 Break-Even Analysis}\label{break-even-analysis}

Explain when the business becomes financially sustainable.

Prompts:\\
- What is the fixed cost baseline?\\
- What level of revenue covers fixed and variable costs?\\
- How many customers or units are needed to break even?

Example table:

\begin{longtable}[]{@{}ll@{}}
\toprule\noalign{}
Component & Value \\
\midrule\noalign{}
\endhead
\bottomrule\noalign{}
\endlastfoot
Fixed Costs & \\
Contribution Margin per Unit & \\
Break-Even Units & \\
\end{longtable}

(Placeholder for break-even graph)

\section{12.6 Cash Flow Forecast}\label{cash-flow-forecast}

Project cash inflows and outflows to ensure liquidity and
sustainability.

Prompts:\\
- What early investments are required?\\
- When does cash flow turn positive?\\
- How much buffer or runway is needed?

Example table:

\begin{longtable}[]{@{}
  >{\raggedright\arraybackslash}p{(\linewidth - 8\tabcolsep) * \real{0.2027}}
  >{\raggedright\arraybackslash}p{(\linewidth - 8\tabcolsep) * \real{0.1351}}
  >{\raggedright\arraybackslash}p{(\linewidth - 8\tabcolsep) * \real{0.1486}}
  >{\raggedright\arraybackslash}p{(\linewidth - 8\tabcolsep) * \real{0.2162}}
  >{\raggedright\arraybackslash}p{(\linewidth - 8\tabcolsep) * \real{0.2973}}@{}}
\toprule\noalign{}
\begin{minipage}[b]{\linewidth}\raggedright
Month/Quarter
\end{minipage} & \begin{minipage}[b]{\linewidth}\raggedright
Cash In
\end{minipage} & \begin{minipage}[b]{\linewidth}\raggedright
Cash Out
\end{minipage} & \begin{minipage}[b]{\linewidth}\raggedright
Net Cash Flow
\end{minipage} & \begin{minipage}[b]{\linewidth}\raggedright
Cumulative Position
\end{minipage} \\
\midrule\noalign{}
\endhead
\bottomrule\noalign{}
\endlastfoot
Q1 & & & & \\
Q2 & & & & \\
\end{longtable}

(Placeholder for cash flow timeline)

\section{12.7 Capital Requirements}\label{capital-requirements}

Describe funding needs, if any.

Prompts:\\
- What capital is required to reach breakeven?\\
- What is the timing of capital needs?\\
- What will the capital be used for?\\
- Are there plans for external investment, grants, or partnerships?

(Placeholder for funding structure diagram)

\section{12.8 Financial Risks}\label{financial-risks}

Identify financial uncertainties and potential impacts.

\begin{longtable}[]{@{}llll@{}}
\toprule\noalign{}
Risk & Likelihood & Impact & Mitigation \\
\midrule\noalign{}
\endhead
\bottomrule\noalign{}
\endlastfoot
Lower-than-expected adoption & & & \\
Higher operational costs & & & \\
Market downturn & & & \\
\end{longtable}

(Placeholder for financial risk matrix)

\section{12.9 Key Financial
Assumptions}\label{key-financial-assumptions}

List assumptions used in building the model.

Examples:\\
- Pricing per tier\\
- Average deal size\\
- Implementation cycle time\\
- Customer churn\\
- Support cost per customer\\
- Hosting and compute cost trends

(Placeholder for assumption summary table)

\section{12.10 KPIs for Financial
Performance}\label{kpis-for-financial-performance}

Define the metrics that measure financial health.

\begin{longtable}[]{@{}lll@{}}
\toprule\noalign{}
KPI & Definition & Target \\
\midrule\noalign{}
\endhead
\bottomrule\noalign{}
\endlastfoot
MRR / ARR & Recurring revenue & \\
CAC & Cost of acquiring one customer & \\
LTV & Lifetime value of a customer & \\
Burn Rate & Monthly net cash usage & \\
Runway & Months of cash left & \\
Gross Margin & Revenue minus direct costs & \\
\end{longtable}

(Placeholder for KPI dashboard)

\chapter{13. Growth Roadmap}\label{growth-roadmap}

\section{13.1 Overview}\label{overview-6}

Provide a structured multi-phase plan for expanding the business across
products, markets, and operations.

Prompts to complete:\\
- What does growth look like over time?\\
- What milestones define success at each stage?\\
- How do product, team, and market evolve?

(Placeholder for roadmap overview diagram)

\section{13.2 Strategic Growth
Objectives}\label{strategic-growth-objectives}

List the primary goals that define the long-term direction of the
business.

Examples:\\
- Expand product capabilities\\
- Enter new markets or customer segments\\
- Increase automation and reduce delivery effort\\
- Build partner ecosystems\\
- Strengthen brand authority in the domain

Table format:

\begin{longtable}[]{@{}lll@{}}
\toprule\noalign{}
Objective & Description & Time Horizon \\
\midrule\noalign{}
\endhead
\bottomrule\noalign{}
\endlastfoot
Objective A & & \\
Objective B & & \\
\end{longtable}

(Placeholder for objectives mapping)

\section{13.3 Phase 1: Foundation (0--12
Months)}\label{phase-1-foundation-012-months}

Describe early-stage priorities and outcomes.

Prompts:\\
- What must be built or validated first?\\
- Which customer segments should be targeted?\\
- What internal capabilities must be established?

Example initiatives: - Build core product functionality\\
- Deliver pilot projects\\
- Gather user feedback\\
- Create onboarding and documentation\\
- Establish minimal GTM engine\\
- Form key partnerships

\begin{longtable}[]{@{}llll@{}}
\toprule\noalign{}
Initiative & Owner & Outcome & Priority \\
\midrule\noalign{}
\endhead
\bottomrule\noalign{}
\endlastfoot
Initiative A & & & \\
Initiative B & & & \\
\end{longtable}

(Placeholder for Phase 1 timeline)

\section{13.4 Phase 2: Acceleration (12--24
Months)}\label{phase-2-acceleration-1224-months}

Focus on scaling early traction into repeatable growth.

Prompts:\\
- How will customer acquisition be scaled?\\
- What product enhancements support more users?\\
- What operational changes prepare the organisation for higher volume?

Possible initiatives: - Expand integrations and modules\\
- Increase automation of workflows\\
- Launch structured training programs\\
- Develop partner-led delivery capabilities\\
- Strengthen brand presence

\begin{longtable}[]{@{}llll@{}}
\toprule\noalign{}
Initiative & Owner & Outcome & Priority \\
\midrule\noalign{}
\endhead
\bottomrule\noalign{}
\endlastfoot
Initiative A & & & \\
Initiative B & & & \\
\end{longtable}

(Placeholder for Phase 2 roadmap)

\section{13.5 Phase 3: Scale and Maturity (24--36 Months and
Beyond)}\label{phase-3-scale-and-maturity-2436-months-and-beyond}

Describe long-term scaling strategies.

Prompts:\\
- What does the organisation look like at maturity?\\
- Which markets or regions will be targeted?\\
- What enterprise requirements must be met?\\
- What advanced capabilities strengthen long-term advantage?

Possible initiatives: - Enterprise-grade compliance and security\\
- Geographic expansion\\
- Full ecosystem of partners and integrators\\
- AI-driven predictive and prescriptive capabilities\\
- Mature self-service onboarding

\begin{longtable}[]{@{}llll@{}}
\toprule\noalign{}
Initiative & Owner & Outcome & Priority \\
\midrule\noalign{}
\endhead
\bottomrule\noalign{}
\endlastfoot
Initiative A & & & \\
Initiative B & & & \\
\end{longtable}

(Placeholder for Phase 3 timeline)

\section{13.6 Product Roadmap
Alignment}\label{product-roadmap-alignment}

Ensure that product evolution aligns with growth phases.

Prompts:\\
- What features are required in each phase?\\
- Which enhancements unlock new segments?\\
- What technical debt must be cleared?

Example:

\begin{longtable}[]{@{}ll@{}}
\toprule\noalign{}
Phase & Key Product Releases \\
\midrule\noalign{}
\endhead
\bottomrule\noalign{}
\endlastfoot
Foundation & \\
Acceleration & \\
Scale & \\
\end{longtable}

(Placeholder for product roadmap graphic)

\section{13.7 Organisational Growth
Requirements}\label{organisational-growth-requirements}

Describe how the team and structure evolve.

Areas to consider: - Hiring plans\\
- New roles (support, engineering, operations, training)\\
- Leadership and governance\\
- Partnerships and outsourcing

(Placeholder for organisation evolution chart)

\section{13.8 Market Expansion}\label{market-expansion}

Identify which new markets, verticals, or customer segments should be
pursued.

Prompts:\\
- Which early markets validate the model?\\
- Which segments offer the fastest revenue?\\
- Which long-term markets offer strategic positioning?

Example:

\begin{longtable}[]{@{}lll@{}}
\toprule\noalign{}
Market / Segment & Entry Timing & Rationale \\
\midrule\noalign{}
\endhead
\bottomrule\noalign{}
\endlastfoot
Segment A & Early & \\
Segment B & Later & \\
\end{longtable}

(Placeholder for market expansion timeline)

\section{13.9 Partnership and Ecosystem
Development}\label{partnership-and-ecosystem-development}

Explain how partnerships will amplify growth over time.

Prompts:\\
- Which partnerships increase distribution?\\
- Which partnerships add credibility?\\
- Which technology alliances expand capabilities?

Examples: - Academic institutions\\
- Technology vendors\\
- Industry associations\\
- Integration partners

(Placeholder for partner ecosystem map)

\section{13.10 Scaling Infrastructure}\label{scaling-infrastructure}

Describe technical and operational scaling requirements.

Prompts:\\
- When does infrastructure need to scale horizontally or vertically?\\
- What monitoring, automation, or testing must be added?\\
- What compliance frameworks become necessary?

(Placeholder for infrastructure scaling plan)

\section{13.11 Key Growth Metrics}\label{key-growth-metrics}

Define the KPIs that measure progress across phases.

\begin{longtable}[]{@{}lll@{}}
\toprule\noalign{}
Metric & Definition & Phase Priority \\
\midrule\noalign{}
\endhead
\bottomrule\noalign{}
\endlastfoot
ARR Growth & Annual recurring revenue & All \\
CAC vs LTV & Acquisition cost vs lifetime value & Acceleration \\
Churn Rate & Customer retention & Scale \\
Product Adoption & Active usage of key features & Foundation \\
Net Margin & Profitability & Scale \\
\end{longtable}

(Placeholder for growth dashboard)

\section{13.12 Risks to Growth and
Mitigation}\label{risks-to-growth-and-mitigation}

Document risks that could slow or block growth.

\begin{longtable}[]{@{}llll@{}}
\toprule\noalign{}
Risk & Impact & Likelihood & Mitigation \\
\midrule\noalign{}
\endhead
\bottomrule\noalign{}
\endlastfoot
Operational bottlenecks & & & \\
Market saturation & & & \\
Increased competition & & & \\
Talent shortages & & & \\
\end{longtable}

(Placeholder for risk map)

\section{13.13 Long-Term Vision}\label{long-term-vision}

Summarise where the organisation aims to be in 3--5+ years.

Prompts:\\
- What does success look like in full maturity?\\
- What is the position in the market?\\
- How has the product ecosystem evolved?\\
- What impact has the organisation achieved?

(Placeholder for vision statement panel)

\chapter{14. Risk Analysis and
Mitigation}\label{risk-analysis-and-mitigation}

\section{14.1 Overview}\label{overview-7}

Identify the strategic, operational, financial, and external risks that
could affect the business, along with mitigation strategies.

Prompts to complete:\\
- What uncertainties threaten success?\\
- What categories do these risks fall into?\\
- How severe and likely are they?\\
- How will the business anticipate, monitor, and respond to each risk?

(Placeholder for risk framework overview)

\section{14.2 Strategic Risks}\label{strategic-risks}

Risks related to the overall direction, competitiveness, or positioning
of the business.

\begin{longtable}[]{@{}lllll@{}}
\toprule\noalign{}
Risk & Description & Impact & Likelihood & Mitigation \\
\midrule\noalign{}
\endhead
\bottomrule\noalign{}
\endlastfoot
Market Misalignment & & & & \\
Rapid Competitor Advances & & & & \\
Technology Obsolescence & & & & \\
Failure to Scale & & & & \\
\end{longtable}

(Placeholder for strategic risk heatmap)

\section{14.3 Operational Risks}\label{operational-risks}

Risks that arise from internal processes, delivery, infrastructure, or
staffing.

\begin{longtable}[]{@{}lllll@{}}
\toprule\noalign{}
Risk & Description & Impact & Likelihood & Mitigation \\
\midrule\noalign{}
\endhead
\bottomrule\noalign{}
\endlastfoot
Delivery Bottlenecks & & & & \\
Support Overload & & & & \\
Process Failures & & & & \\
Infrastructure Outage & & & & \\
Knowledge Silos & & & & \\
\end{longtable}

(Placeholder for operational risk workflow)

\section{14.4 Financial Risks}\label{financial-risks-1}

Risks affecting revenue, cash flow, funding, or cost stability.

\begin{longtable}[]{@{}lllll@{}}
\toprule\noalign{}
Risk & Description & Impact & Likelihood & Mitigation \\
\midrule\noalign{}
\endhead
\bottomrule\noalign{}
\endlastfoot
Lower-than-expected Sales & & & & \\
Rising Infrastructure Costs & & & & \\
Customer Churn & & & & \\
Budget Cuts in Target Sectors & & & & \\
\end{longtable}

(Placeholder for financial scenario modeling)

\section{14.5 Market and External
Risks}\label{market-and-external-risks}

Risks driven by regulatory, economic, or environmental changes.

\begin{longtable}[]{@{}lllll@{}}
\toprule\noalign{}
Risk & Description & Impact & Likelihood & Mitigation \\
\midrule\noalign{}
\endhead
\bottomrule\noalign{}
\endlastfoot
Regulatory Changes & & & & \\
Data Privacy Requirements & & & & \\
Economic Downturn & & & & \\
Supply Chain Instability & & & & \\
\end{longtable}

(Placeholder for external risk map)

\section{14.6 Client and Stakeholder
Risks}\label{client-and-stakeholder-risks}

Risks emerging from client dependencies, expectations, or relationships.

\begin{longtable}[]{@{}lllll@{}}
\toprule\noalign{}
Risk & Description & Impact & Likelihood & Mitigation \\
\midrule\noalign{}
\endhead
\bottomrule\noalign{}
\endlastfoot
Misaligned Expectations & & & & \\
Client Dependency & & & & \\
Slow Decision Cycles & & & & \\
Data Quality Issues & & & & \\
\end{longtable}

(Placeholder for stakeholder risk matrix)

\section{14.7 Technical and Security
Risks}\label{technical-and-security-risks}

Risks arising from the architecture, integrations, or data security.

\begin{longtable}[]{@{}lllll@{}}
\toprule\noalign{}
Risk & Description & Impact & Likelihood & Mitigation \\
\midrule\noalign{}
\endhead
\bottomrule\noalign{}
\endlastfoot
Integration Failures & & & & \\
Data Breach & & & & \\
Model or Algorithm Errors & & & & \\
Vendor Lock-in & & & & \\
\end{longtable}

(Placeholder for security risk diagram)

\section{14.8 Legal and Compliance
Risks}\label{legal-and-compliance-risks}

Risks related to contracts, intellectual property, or regulatory
frameworks.

\begin{longtable}[]{@{}lllll@{}}
\toprule\noalign{}
Risk & Description & Impact & Likelihood & Mitigation \\
\midrule\noalign{}
\endhead
\bottomrule\noalign{}
\endlastfoot
Contract Disputes & & & & \\
IP Ownership Conflicts & & & & \\
Non-Compliance with Standards & & & & \\
\end{longtable}

(Placeholder for compliance controls)

\section{14.9 Risk Monitoring
Framework}\label{risk-monitoring-framework}

Define how risks will be continually monitored and reviewed.

Prompts:\\
- Who owns risk identification and oversight?\\
- How often are risks reviewed?\\
- What thresholds trigger escalation?\\
- What reporting mechanisms exist?

Example categories: - Weekly operational reviews\\
- Monthly leadership check-ins\\
- Quarterly risk audits\\
- Automated alerts for technical indicators

(Placeholder for monitoring workflow)

\section{14.10 Scenario Planning}\label{scenario-planning}

Outline potential high-impact scenarios and planned responses.

Examples: - Sudden surge in demand → capacity plan\\
- Major system outage → disaster recovery plan\\
- Regulatory shift → compliance review\\
- Key customer loss → revenue diversification plan

(Placeholder for scenario matrix)

\section{14.11 Residual Risk Assessment}\label{residual-risk-assessment}

After mitigation measures, evaluate remaining risks.

\begin{longtable}[]{@{}llll@{}}
\toprule\noalign{}
Risk & Residual Impact & Residual Likelihood & Acceptable? \\
\midrule\noalign{}
\endhead
\bottomrule\noalign{}
\endlastfoot
Example Risk A & & & \\
Example Risk B & & & \\
\end{longtable}

(Placeholder for residual risk chart)

\section{14.12 Summary of Mitigation
Strategy}\label{summary-of-mitigation-strategy}

Summarise the overarching philosophy or system for risk control.

Prompts:\\
- What principles guide risk decisions?\\
- How do mitigation actions align with long-term objectives?\\
- How are high-impact risks prioritised?

(Placeholder for mitigation strategy overview)


\backmatter


\end{document}
